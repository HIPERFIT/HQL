\documentclass[11pt,a4paper]{article}

%---- defitions ----
\def\Author{
\bf https://github.com/HIPERFIT/HQL
}
\def\Title{\bf Hiperfit Quant Library \\ Documentation}

\usepackage{tikz}
\usepackage{gnuplottex}
\usepackage[]{amsmath}
\usepackage{amssymb}
\usepackage[english]{babel}
\usepackage[utf8]{inputenc}
\usepackage{graphicx}
\usepackage{moreverb}
\usepackage{hyperref}
\usepackage{color}
\usepackage{listings,setspace,framed}
\usepackage{pgfplots}
%\usepackage{bashful}
\usepackage{tikz}
\usetikzlibrary{decorations.pathreplacing}
\usepackage{lmodern,inconsolata}
\usepackage{array, xcolor, lipsum, bibentry, fancyhdr}
\usepackage[absolute]{textpos}
\usepackage[top=25mm, bottom=25mm, left=22mm, right=22mm]{geometry} %Layout of page
\usepackage{lastpage} % number of last page 

%\usepackage[table]{xcolor}
%\usepackage[table]{xcolor}
%\usepackage{xcolor}
%\usepackage{filter}
%\usepackage{vim}
%\usepackage{caption}
\usepackage[compatibility=true]{caption} 
\usepackage{etex}
\usepackage{ctable}
\captionsetup[figure]{labelfont=bf}
\captionsetup[table]{labelfont=bf,position=below}

\numberwithin{equation}{section}

%---- settings ----
% Comments
\newcommand{\comm}[2]{{\sf \(\spadesuit\){\bf #1: }{\rm \sf #2}\(\spadesuit\)}}
\newcommand{\mcomm}[2]{\marginpar{\scriptsize \comm{#1}{#2}}}
\newcommand{\ab}[1]{\mcomm{AB}{#1}}
\newcommand{\ja}[1]{\mcomm{JA}{#1}}

%\pagestyle{fancy}

\renewcommand{\headrulewidth}{0.5pt}
\renewcommand{\footrulewidth}{1pt}

%% Font size of tables
%\let\oldtabular\tabular
%\renewcommand{\tabular}{\small\oldtabular}

\usepackage[T1]{fontenc} % font
\setlength{\parindent}{0in}
\definecolor{lightgray}{rgb}{0.9,0.9,0.9}

\newenvironment{filecode}[1][]
{\minipage{\linewidth}
\lstset{basicstyle=\ttfamily\footnotesize,frame=single,
numberstyle=\small\color{black},keywordstyle=\color{black},commentstyle=\color{black},
stringstyle=\color{black},tabsize=2,backgroundcolor=\color{lightgray},language=Haskell,#1}}
{\endminipage}
\renewcommand*\rmdefault{ppl}

\lfoot{
\begin{textblock*}{100mm}(30mm, 280mm )
\end{textblock*}
}

\pagestyle{fancy}
\fancyhf{} 
 
\lhead{\uppercase{Hiperfit Quant Library}}
\rhead{\nouppercase{\rightmark}}
 
\cfoot{\thepage\ / \phantomsection\pageref*{LastPage}}
 
\lfoot{
\begin{textblock*}{100mm}(30mm, 280mm )
\end{textblock*}
}

%\newcolumntype{C}{>{\small}c}

%-------------------

\newcommand{\tikzmark}[1]{\tikz[overlay,remember picture] \node (#1) {};}
\newcommand{\DrawBox}[1][]{%
    \tikz[overlay,remember picture]{
    \draw[red,#1]
      ($(left)+(-0.2em,0.9em)$) rectangle
      ($(right)+(0.2em,-0.3em)$);}
}

\definecolor{comments}{rgb}{0,0.6,0}
\definecolor{strings}{rgb}{0.9,0,0}
\definecolor{keywords}{rgb}{0,0,0}
\definecolor{identifier}{rgb}{0.10,0.10,0.10}
\definecolor{background}{rgb}{0.98,0.98,0.98}


\lstdefinestyle{BashInputStyle}{
  language=haskell,
  breaklines=true,
  breakatwhitespace=true,
  frame=tblr,
  columns=fullflexible,
  backgroundcolor=\color{background},
  linewidth=1.0\linewidth,
  xleftmargin=0.1\linewidth,
  keywordstyle= \color{keywords},
  identifierstyle=\color{identifier},
  stringstyle=\color{strings},
  commentstyle=\color{comments},
  basicstyle=\fontsize{10}{12}\color{black}\ttfamily    
}

\lstdefinestyle{Output}{
  language=haskell,
  breaklines=true,
  breakatwhitespace=true,
  frame=tblr,
  columns=fullflexible,
  backgroundcolor=\color{gray!10},
  linewidth=1.0\linewidth,
  xleftmargin=0.1\linewidth,
  keywordstyle= \color{keywords}\bfseries,
  identifierstyle=\color{identifier},
  stringstyle=\color{strings},
  commentstyle=\color{comments},
  basicstyle=\fontsize{10}{12}\color{black}\bfseries
}

\lstset{escapechar=@,style=BashInputStyle}

\begin{document}

\title{HQL Documentation}

\title{\Title}
\author{\Author}
\date{\today}
\maketitle

\tableofcontents

\vskip 0.5\FrameSep

\section{Introduction}
This is the documentation of the Hiperfit Quant Library, HQL. As a reader you will
find valuable information about the theory behind the implementation together with
illustrative examples of real world use cases. The documentation itself is \textit{Literate Haskell}\cite{LitHaskell},
which means you can load this file into GHCi:

\vskip 0.5\FrameSep
\begin{lstlisting}
~/HQL$ ghci docs/hql.lhs
\end{lstlisting}

\section{Terminology}
In this section the terminology used in the field of fixed income and how it applies to HQL.

\ctable[
caption = Terminology,
pos = ht,
doinside=\small
]{ll}{
}{
  \toprule
    Input & Meaning \\
  \midrule
    Settlement & Settlement date \\
    Maturity & Maturity date \\
    Period & Coupon payment period \\ 
    Basis & Day-count basis \\ 
    EndMonthRule & End-of-month payment rule \\ 
    IssueDate & Bond issue date \\ 
  \bottomrule
}

\section{Compounding}
\subsection{Discrete}
\subsubsection{Linear}

$FV$, the future value in $T$ periods of a nominal $N$, is expressed as:

\[
FV_T = N(1+R_TT)
\]

where $R_T$ is the interest rate.

\subsubsection{Exponential}

Future value of a nominal cash flow $N$ under exponential compounding over
$n$ periods:

\[ FV_T = N(1+R_T)^n \]

The present value of a nominal future cash flow $N$ is therefore:

\[ PV_T = \frac{N}{(1+R_T)^n} \]

Periodic compounding where $n$ is the number of times interest compounded per period

\[ 
FV_T=N\left( 1 + \frac{R}{n}  \right)^{nT}
\]

\subsection{Continuous}
Continuous compounding is defined as taking the limit of the periodic compounding as n goes to infinity,
\[ 
FV=\lim_{n \to \infty} N \left(1+\frac{R}{n}\right)^{nT}=Ne^{RT}
\]

\subsection{Transition between discrete and continuous}
Table of interest rates at various compounding frequencies.

\ab{Jost said we need to describe the table here}

\ctable[
caption = Interest rates at various compounding frequencies,
pos = ht,
doinside=\small
]{lllllll}{
}{
\toprule
 Nominal Rate &  Semi-Annual & Quarterly &  Monthly &  Weekly &  Daily  & Continuous \\
  \midrule
    1.0000       & 1.0025       & 1.0038     & 1.0046   & 1.0049  & 1.0050  & 1.0050      \\
    2.0000       & 2.0100       & 2.0151     & 2.0184   & 2.0197  & 2.0201  & 2.0201      \\
    5.0000       & 5.0625       & 5.0945     & 5.1162   & 5.1246  & 5.1267  & 5.1271      \\
    10.0000      & 10.2500      & 10.3813    & 10.4713  & 10.5065 & 10.5156 & 10.5171     \\
    15.0000      & 15.5625      & 15.8650    & 16.0755  & 16.1583 & 16.1798 & 16.1834     \\
    20.0000      & 21.0000      & 21.5506    & 21.9391  & 22.0934 & 22.1336 & 22.1403     \\
    25.0000      & 26.5625      & 27.4429    & 28.0732  & 28.3256 & 28.3916 & 28.4025     \\
    50.0000      & 56.2500      & 60.1807    & 63.2094  & 64.4788 & 64.8157 & 64.8721     \\ 
  \bottomrule
}

\subsection{Calculating present and future value}
%To calculate the present value of a future cash flow we use the discount factor
%to scale the amount:

%We can calculate the future value by using the inverse of the discount factor above:

To calculate present and future value in HQL, two objects are of particular interest:
the DiscountFactor and the CompoundFactor. Let's start by using the DiscountFactor in an example
where we have a monthly compounded interest rate of 8.5\% during a period of 3 years:
\vskip 0.5\FrameSep
\begin{lstlisting}
HQL> let df = discountFactor (InterestRate (Periodic 12) 8.5) 3.0 0.0
\end{lstlisting}
We can use the previously defined DiscountFactor to discount an arbitrary amount. Let's say
\$1000.00:
\vskip 0.5\FrameSep
\begin{lstlisting}
HQL> df*1000.0
\end{lstlisting}
yields the following output:
\vskip 0.5\FrameSep
\begin{lstlisting}[style=Output]
775.6133702070988
\end{lstlisting}

\subsection{The continuous discount factor}
The \textit{continuous discount factor} with maturity $T$ at time $t$ is defined as,
\[
p(t,T)=e^{-r(T-t)}; T>t
\]
Which equals the \textit{zero-coupon unit bond}, with face $N=1$. The discount factor
can be seen as entities in a vector, the \textit{discount function},

\[
\mathbf{df} = (p(t,T_1), ..., p(t,T_n))
\]

In the same way, cash flows can be considered entities in a vector,

\[
\mathbf{cf} = (c_1, ..., c_i)
\]

For example, if we were to calculate the present value of a series of future cash flows,
\[
\mathbf{cf} = (\$1000.00,\$1500.00,\$1000.00,\$2000.00)
\]
We can multiply by the discount function with $T_n=n$ years and the rate at 5.00\% continuously compounded,
\[
p(0) = \mathbf{df}\cdot\mathbf{cf} = (e^{-rT_1},...,e^{-rT_n})\cdot(p(0,T_1), ...,p(0,T_4))=
\]
\[
(e^{-0.05\cdot1},e^{-0.05\cdot2},e^{-0.05\cdot3},e^{-0.05\cdot4}) \cdot (\$1000.00,\$1500.00,\$1000.00,\$2000.00)=
\]
\[
(\$951.23,\$1357.26,\$860.71,\$1637.46)
\]

The above calculation is done in HQL using the function \textit{zipWith} taking the discount function and the list of future cash flows as arguments.
\vskip 0.5\FrameSep
\begin{lstlisting}
HQL> zipWith (*) (discountFactors (interestRate Continuous 5.0) 5 4 <*> pure 0) [1000,1500,1000,2000]
\end{lstlisting}
\vskip 0.5\FrameSep
yields the following output:
\vskip 0.5\FrameSep
\begin{lstlisting}[style=Output]
[951.229424500714,1357.2561270539393,860.7079764250578,1637.4615061559637]
\end{lstlisting}
\vskip 0.5\FrameSep
A slightly modified version will return the sum of the cash flows. This is the basis of bond pricing, which is covered in section \nameref{sec:fi}.
\vskip 0.5\FrameSep
\begin{lstlisting}
HQL> sum $ zipWith (*) (discountFactors (interestRate Continuous 5.0) 5 4 <*> pure 0) [1000,1500,1000,2000]
\end{lstlisting}
\vskip 0.5\FrameSep
yields the following output:
\vskip 0.5\FrameSep
\begin{lstlisting}[style=Output]
4806.655034135675
\end{lstlisting}


\section{Interest rates}
\ctable[
caption = The caption is centered by default,
pos = h,
doinside=\small,
center,
]{llll}{
}{
\toprule
    Maturity (months) & Spot rate & Forward & Rate \\
  \midrule
    3 & 4.5\% & $F_{0,3}$ & 4.5\% \\
    6 & 4.3\% & $F_{3,3}$ & 4.05\% \\
    9 & 4.2\% & $F_{6,3}$ & 3.92\% \\
    12 & 4.0\% & $F_{9,3}$ & 3.3\% \\
  \bottomrule
}

\subsection{Spot rates}
\subsection{Forward rates}
Forward rates indicate the interest rate between two future dates [S,T]
\[
(1+R_T)^T=(1+R_S)^S(1+R(t;S,T))^{T-S}; T>S
\]
Solving for the forward rate yields
\[
R(t=0,S,T)=\left( \frac{(1+R_T)^T}{(1+R_S)^S} - 1 \right)^{\frac{1}{T-S}};T>S
\]

Continuous compounding version of the above

\[
e^{R_TT}=e^{R_SS}e^{R(t;S,T)(T-S)};T>S
\]
Solving for the forward rate
\[
R(t;S,T)=\frac{R_TT-R_SS}{T-S}
\]

Let's look at an example of how to determine the forward rate given two time periods and interest rates. The input parameters are listed in the table below.

\ctable[
caption = Input parameters to example,
pos = ht,
width=80mm,
center,
doinside=\small
]{ll}{
}{
  \toprule
    Parameter & Value \\ 
  \midrule
    Compounding & Semi-annually ($n=2$)\\
    $\tau_1$ & 0.25\\
    $\tau_2$ & 1.5 \\
    $R_1$ & 6.65 \\
    $R_2$ & 7.77 \\
  \bottomrule
}

\vskip 0.5\FrameSep
\begin{lstlisting}
HQL> (((discountFactor (InterestRate (Periodic 2) 6.65) 0.25 0.0)/(discountFactor (InterestRate (Periodic 2) 7.77) 1.5))**(1/(2*1.25))-1)*2*100
\end{lstlisting}
yields the following output:
\vskip 0.5\FrameSep
\begin{lstlisting}[style=Output]
7.994727369824384
\end{lstlisting}

\subsection{LIBOR}
The LIBOR (London InterBank Offered Rate) interest rate is the most important interbank rate
used for fixed income valuation. LIBOR rates are quoted on a simple compounding basis with maturities
from overnight to 12 months.
\\
\\
Table of LIBOR rates per 2013-12-26,

\ctable[
caption = LIBOR rates 2013-12-26,
pos = ht,
width=60mm,
center,
doinside=\small
]{ll}{
}{
  \toprule
    Period & LIBOR \\
  \midrule
    1 month & 0.17\% \\ 
    3 month & 0.24\% \\
    6 month & 0.35\% \\
    12 month & 0.59\% \\
  \bottomrule
}


%$R$ is the annual rate
%The value after m compounding periods with n compounding periods per year

\[ 
1+(T-S)L=\frac{p(t,S)}{p(t,T)}
\]
\[ 
e^{r(T-S)}=\frac{p(t,S)}{p(t,T)}
\]
\\
\textbf{LIBOR forward rate}
Simply-compounded forward rate $[S,T]$ prevailing at time $t$,
\[
L(S,T) = -\frac{p(S,T)-p(t,S)}{(T-S)p(t,T)}
\]
\\
\textbf{LIBOR spot rate},

\[
L(S,T)=\frac{p(S,T)-1}{(T-S)p(S,T)}
\]
\\
\textbf{LIBOR forward rate},
Continuously compounded forward rate $[S,T]$ prevailing at time $t$,

\[
R(t;S,T)=-\frac{\log{p(t,T)}-\log{p(t,S)}}{T-S}
\]
\\
\textbf{LIBOR spot rate},
\[
R(S,T)=\frac{\log{p(S,T)}}{T-S}
\]
\\
\textbf{Instantaneous forward rate},
\[
f(t,T)=-\frac{\partial\log{p(t,T)}}{\partial{T}}
\]
\\
\textbf{Instantaneous short rate},
\[
r(t)=f(t,t)
\]

For example, the three-month forward LIBOR for the period $[S,T]$, where $T=S+\tau$ and $\tau=1/4$,
\[
L(t,T) = F(t;S,S+\tau) = F(t;S,T), T>S
\]

\subsubsection{Calculating LIBOR rates}

\section{Interest rate Swaps}
\subsection{Theory and valuation}
\subsection{Cash flows}
\section{Bonds}
\label{sec:fi}

\subsection{Zero coupon bond}
Zero coupon bond and the discount factor

\subsection{Fixed coupon bond}
Series of payments called coupons and a nominal value N (face value)
A number of future payment dates $T_1 < ... < T_n$ where is the maturity of the bond
A sequence of deterministic coupons $c_1, ..., c_n$

\[ c_{T_i} = \left\{
\begin{array}{ll}
  Nc, & \text{for } i=1,...,n-1,  \\
  N(c+1), &\text{for } i=n.
\end{array} \right.\]

\[
p(t) = N \cdot p(t,T_n)+\sum_{i=1}^{n} (c_ip(t,T_i))
\]
From above, we can see that the fixed coupon bond can be reconstructed by a portfolio of zero
bonds (scaled by the coupon), $p(t,T)$ is a zero bond with N=1.
\[
p(t)=\left[p(t,T_n)+r\delta\sum_{i=1}^{n} p(t,T_i)\right]\cdot N
\]
Where $r$ is the coupon rate. For a standardized coupon bond the time intervals will be equally
spaced according to
\[
T_i=T_0+i\delta
\]

Convert above to a list of payments at offset times. Where denotes the time interval in years
according to the day-count convention, see below.

\subsubsection{Example - Calculating present value}

\ctable[
caption = Input parameters to example,
pos = h,
center,
doinside=\small
]{llllll}{
}{
  \toprule
    Maturity (years) & 1 & 2 & 3 & 4 & 5\\ 
  \midrule
    Interest rate & 3.00\% & 3.25\% & 3.50\% & 3.55\% & 3.30\%\\
  \bottomrule
}

Let's consider a bond maturing in 5 years, paying a coupon of 10\% and a face of 1000,
\ctable[
caption = Bond price,
pos = h,
center,
doinside=\small
]{lllll}{
}{
  \toprule
    Maturity & Interest rate & Cash flow & Present value \\
  \midrule
    1 & 3.00\% & 100 & 97.09 \\
    2 & 3.25\% & 100 & 93.80 \\
    3 & 3.50\% & 100 & 90.19 \\
    4 & 3.55\% & 100 & 86.98 \\
    5 & 3.30\% & 1100 & 935.17 \\
\midrule
    Price bond  &     &  & \textbf{1303.23} \\
  \bottomrule
}

\framebox{TODO: Adapt HQL to handle this case}

\subsection{Floating coupon bond}
Not yet supported by HQL (this means stochastic interest rates).

\section{Standard bond types}
\subsection{Annuity}

\subsubsection{Example}
Suppose we want to price an annuity maturing in 5 years, a face of \$1000.00 at an annual interest rate of 5.00\%.
This can be done in continuous time integrating the continuous discount factor $e^-rt$:
\[
p(0)=N\cdot \int_{t=0}^T \mathrm{e}^{-rt}\,\mathrm{d}t=\frac{N}{R}\cdot (e^{-rt}-e^{-rT})=\frac{N}{R}(1-e^{-rT})
\]
Plugging in the numbers from above gives:
\[
p(0)=\frac{\$1000.00}{0.05}(1-e^{-\log(1+0.005)\cdot 5})\approx \$4329.48
\]
\begin{lstlisting}
HQL> df*1000.0
\end{lstlisting}
yields the following output:
\vskip 0.5\FrameSep
\begin{lstlisting}[style=Output]
775.6133702070988
\end{lstlisting}

\subsection{Bullet}
Bullet has a fixed rate coupon which is paid at every settlement. No repayments before maturity.

\[ p(t,T) = \sum_{t=1}^{T}\frac{C}{(1+r)^t} + \frac{N}{(1+r)^T} \]

\subsection{Consol}
Consol has a fixed rate coupon and never terminates and there is no payments at any settlement,
only interest (the coupon) is paid.

\[
p(t) = N \cdot p(t,T_n)+\sum_{i=1}^{n} (c_i \cdot p(t,T_i))
\]
\[
\lim_{x \to \infty} p(t,\infty)=\left[r\delta\sum_{i=1}^{\infty} p(t,T_i)\right]\cdot N = 
\sum_{i=0}^{\infty} p(t,T_i) \cdot \frac{\delta N}{e^{rT}} \cdot \to \frac{\delta N}{e^{rT}}\left[ \frac{e^{rT}}{r}\right] = \frac{\delta N}{r}
\]
Which means the present value of the consol bond is represented by the the following formula:
\[
p(t) = \frac{\delta N}{r}
\]

\framebox{p(t) function of time?}

\subsection{Serial}
Serial has a fixed rate coupon, the repayments are spread evenly over all remaining settlements and the coupon payments, $C_t$ decline
over time.
\[ p(t,T) = \sum_{t=1}^{T}\frac{C_t}{(1+r)^t} + \frac{N}{(1+r)^T} \]

\begin{figure}[h!]
\begin{center}
\begin{tikzpicture}[-,shorten >=1pt,auto,node distance=1.5cm,thick,minimum size=0.8cm,main node/.style={circle,draw=red,very thick}]
\tikzstyle{selected edge} = [draw,line width=6pt,-,blue!30]

\coordinate (belowstart) at (0,-1);
\coordinate (start) at (0,0);
\coordinate (stop) at (10.2,0);
\coordinate (abovestop) at (10.2,2);
\draw (start) -- (stop);
\draw (start) -- (belowstart);
\draw (stop) -- (abovestop);

\node at (0, -1.5) () {$t=0$};

\filldraw[draw=black, fill=blue] (2,0) rectangle node {} +(0.2,1);
\filldraw[draw=black, fill=red]  (2,1) rectangle node {} +(0.2,1);
\node at (2, 2.5) () {$t=2$};

\filldraw[draw=black, fill=blue] (4,0) rectangle node {} +(0.2,1);
\filldraw[draw=black, fill=red]  (4,1) rectangle node {} +(0.2,0.75);
\node at (4, 2.5) () {$t=4$};

\filldraw[draw=black, fill=blue] (6,0) rectangle node {} +(0.2,1);
\filldraw[draw=black, fill=red]  (6,1) rectangle node {} +(0.2,0.5);
\node at (6, 2.5) () {$t=6$};

\filldraw[draw=black, fill=blue] (8,0) rectangle node {} +(0.2,1);
\filldraw[draw=black, fill=red]  (8,1) rectangle node {} +(0.2,0.25);
\node at (8, 2.5) () {$t=8$};

\filldraw[draw=black, fill=blue] (10,0) rectangle node {} +(0.2,1);
\filldraw[draw=black, fill=red]  (10,1) rectangle node {} +(0.2,0.1);
\node at (10, 2.5) () {$t=10$};

% Legend
\filldraw[draw=black, fill=blue] (7,-1) rectangle node {} +(0.2,0.2);
\node at (8.3, -0.92) () {repayment};

\filldraw[draw=black, fill=red] (7,-1.5) rectangle node {} +(0.2,0.2);
\node at (8, -1.44) () {coupon};

\end{tikzpicture}
\caption{Cashflow of a serial.}
\label{fig:anc}
\end{center}
\end{figure}


\framebox{TODO: Continous time and HQL}

\section{US Treasury Bonds}
Treasury securities are the debt financing instruments of the United States federal government, and they are often referred to simply as Treasuries.
\subsection{Treasury Bill - T-Bill}
Short term debt instruments that mature in one year or less.
\subsection{Treasury Note - T-Note}
Medium term debt instruments that mature in two to ten years.
\subsection{Treasury Bond - T-Bond}
Long term debt instruments that mature in twenty to thirty years.

\subsection{Example using HQL}

\section{The term structure}
The term structure or the yield curve.
Created from zero coupon bonds, zero yield curve
\[
y(t,T)=-\frac{\log{p(t,T)}}{T-t}
\]
The yield curve is constructed by fixating $t$,
\[
T\to y(t,T)
\]
To obtain the discount factor from a yield curve at time $t$,
\[
e^{y(t,T)(T-t)}=e^{-\log{p(t,T)}}\iff e^{y(t,T)(T-t)}=\frac{1}{p(t,T)}
\]
\[
\iff p(t,T)=e^{-y(t,T)(T-t)}
\]
We can price a bond from the yield curve, given face value $N$,
\[
p(t,T)=e^{-y(t,T)(T-t)}\cdot N
\]

\subsection{Flat term structure}
\begin{lstlisting}
HQL> let termStructure x = 5 + (1/2)*sqrt(x)
HQL> discountFactor (InterestRate (Periodic 2) (termStructure 1.5)) 1.5 0.0
\end{lstlisting}

yields the following output:
\vskip 0.5\FrameSep

\begin{lstlisting}[style=Output]
0.9203271932613013
\end{lstlisting}

\subsection{Analytic term structure}

\subsection{Interpolated term structure}
Construct a term structure from bond prices, example

\subsubsection{Example - Treasury term structure}
In this example we are going to create a term structure form the data available for
US treasury bonds. Below is a table with data from 2013-12-26.

\ctable[
caption = US Treasury data 2013-12-26,
pos = ht,
width=80mm,
center,
doinside=\small
]{ll}{
}{
  \toprule
    Instrument & Yield \\
  \midrule
    3-month treasury bill & 0.07\% \\
    1-year treasury bill & 0.13\% \\
    2-year treasury note & 0.42\% \\
    5-year treasury note & 1.74\% \\
    10-year treasury note & 3\% \\
    30-year treasury bond & 3.92\% \\
  \bottomrule
}

\section{Yield curve fitting}
Various methods of fitting a yield curve to data (LIBOR, forex futures, bonds).

\begin{lstlisting}
HQL> df*1000.0
\end{lstlisting}
yields the following output:
\vskip 0.5\FrameSep
\begin{lstlisting}[style=Output]
775.6133702070988
\end{lstlisting}

\subsection{Yield to maturity}

The yield of a bond is the discount rate that makes the future value of all future
cashflows equal to the present value. For coupon payments $C_i$, bond price $N_t^T$
at time $t$ with maturity $T$, the yield $y_t$ satisfies: 

\[
N_t^T = \sum_{T_i > t} C_i (1+y_t)^{-(T_i - t)}, \hspace{2cm} \forall\; T_i < T
\]

For zero-coupon bonds, we therefore have the following for a maturity $T$ at
time $t$:

\[
N_t^T = (1+y_t^T)^{-(T-t)}
\]

with yield, $y_t^T$:

\[
y_t^T = \sqrt[T-t]{N_t^T} - 1
\]

\ab{Note on perpetual bonds}

\subsection{Duration and convexity}

\section{Day-count conventions}
Day-count convention or year fraction, time between two dates t and T.

\framebox{HQL example usage goes here}

\ctable[
caption = Input parameters to example,
pos = ht,
width = 80mm,
center,
doinside=\small
]{lll}{
}{
  \toprule
    Convention & Time interval & $\delta$ \\
  \midrule
    Actual/365 & Annually & 1.0 \\ 
    180/360 & Semi-annually & 0.5 \\ 
    90/360 & Quarterly & 0.25 \\ 
    30/360 & Monthly & 1/12 \\
  \bottomrule
}

Time is measured in years. Day-count conventions describe the time measurement between two
dates t and T. Dates are expressed by (day, month, year). 

\[ \tau = \boldsymbol{T} - \boldsymbol{t};  \left\{
  \begin{array}{l l}
      \boldsymbol{t} = (Y_1, M_1, D_1) \\
\boldsymbol{T}=(Y_2,M_2,D_2)
  \end{array} \right.\]


\subsection{Actual/365}
A year has 365 days, and the days are counted normally:
\[
\delta = \frac{\Delta \tau_D}{365} = \frac{D_2-D_1}{365} = \frac{\text{actual days between } \boldsymbol{t} \text{ and } \boldsymbol{T}}{365}
\]
\subsection{Actual/360}
Same as above except one year is said to be 360 days. Standard for US dollars
\[
\delta = \frac{\Delta \tau_D}{360} =\frac{D_2-D_1}{360} = \frac{\text{actual days between } \boldsymbol{t} \text{ and } \boldsymbol{T}}{360}
\]
\subsection{30/360}
Month are 30 days long, a year is 360 days long
\[
\delta = \frac{360 \cdot \Delta \tau_Y + 30 \cdot \Delta \tau_M + \Delta \tau_D}{360} =\frac{360(Y_2-Y_2)+30(M_2-M_1)+(D_2-D_1)}{360}
\]

\subsection{Business day conventions}
A business day convention is a convention for adjustment of dates when a specified date is not a good business day. 
\begin{description}
  \item[Following] \hfill \\
The adjusted date is the following good business day \ldots
  \item[Preceding] \hfill \\
    The adjusted date is the preceding good business day \ldots
  \item[Modified following] \hfill \\
  The adjusted date is the following good business day unless the day is in the next calendar month \ldots
  \item[End of month] \hfill \\
Where the start date of a period is on the final business day of a particular calendar month \ldots
\end{description}

\newpage

% REFERENCES
\bibliographystyle{abbrv}
%\addcontentsline{toc}{section}{References}
\bibliography{hql}

\end{document}
