\documentclass[11pt,a4paper]{article}

%---- defitions ----
\def\Author{
\bf https://github.com/HIPERFIT/HQL
}
\def\Title{\bf HQL Documentation}

\usepackage{tikz}
\usepackage{gnuplottex}
\usepackage[]{amsmath}
\usepackage{amssymb}
\usepackage[english]{babel}
\usepackage[utf8]{inputenc}
\usepackage{graphicx}
\usepackage{moreverb}
\usepackage{hyperref}
\usepackage{color}
\usepackage{framed}
\usepackage{listings,setspace,framed}
\usepackage{pgfplots}
%\usepackage{bashful}
\usepackage{color}
\usepackage{tikz}
\usetikzlibrary{decorations.pathreplacing}
\usepackage{lmodern,inconsolata}
%\usepackage{float}
\usepackage{array, xcolor, lipsum, bibentry, fancyhdr}
\usepackage[absolute]{textpos}
\usepackage[top=25mm, bottom=25mm, left=22mm, right=22mm]{geometry} %Layout of page
\usepackage{lastpage} % number of last page
\usepackage[titletoc,toc,title]{appendix}
%\usepackage[table]{xcolor}
%\usepackage[table]{xcolor}
%\usepackage{xcolor}
%\usepackage{filter}
%\usepackage{vim}
%\usepackage{mdframed}
\usepackage{caption}
\usepackage[compatibility=true]{caption}
\usepackage{etex}
\usepackage{ctable}
\usepackage{minted}
\usepackage{setspace}
\usepackage{etoolbox}
\usepackage[section]{placeins}
\usepackage{makeidx}
\usepackage{float}
%\usepackage[toc]{glossaries}
\usepackage[toc,acronym]{glossaries} 
%\usepackage[xindy]{glossaries}
%\usepackage[nomain,acronym,toc]{glossaries} 
\usepackage{nomencl}

\newcommand\Loadedframemethod{default}
\usepackage[framemethod=\Loadedframemethod]{mdframed}

\surroundwithmdframed[middlelinecolor=ltxmdfblue,middlelinewidth=1pt,%
                      roundcorner=10pt,innertopmargin=0pt,%
                      leftmargin=1cm,rightmargin=1cm,%
                      innerleftmargin=-15pt,innerrightmargin=-15pt,%
                      ignorelastdescenders,%
                      settings={\lstset{resetmargins}},%
                      skipbelow=\topskip,skipabove=\topskip,%
                      innerbottommargin=0pt,backgroundcolor=gray!10]%
                     {tltxmdfexample}

\newmdenv[middlelinecolor=ltxmdfblue,middlelinewidth=1pt,%
                      roundcorner=10pt,innertopmargin=0pt,%
                      leftmargin=1cm,rightmargin=1cm,%
                      innerleftmargin=-15pt,innerrightmargin=-15pt,%
                      ignorelastdescenders,%
                      settings={\lstset{resetmargins}},%
                      skipbelow=\topskip,skipabove=\topskip,%
                      innerbottommargin=0pt,backgroundcolor=gray!10]%
                     {tltxmdfhighlight}
\def\highlightinputenv{tltxmdfhighlight}


\captionsetup[figure]{labelfont=bf}
\captionsetup[table]{labelfont=bf,position=below}
%\usepackage[bottom]{footmisc}
\numberwithin{equation}{section}

\makeindex 
\makeglossaries

%%% GLOSSARY

\newglossaryentry{Settlement}{name={Settlement}, description={Settlement date}}
\newglossaryentry{Maturity}{name={Maturity}, description={Maturity date}}
\newglossaryentry{Period}{name={Period}, description={Coupon payment period}}
\newglossaryentry{Basis}{name={Basis}, description={Day-count basis}}
\newglossaryentry{EndMonthRule}{name={EndMonthRule}, description={End-of-month payment rule}}
\newglossaryentry{IssueDate}{name={IssueDate}, description={Bond issue date}}
\newglossaryentry{Term}{name={Term}, description={\ldots}}
\newglossaryentry{Tenor}{name={Tenor}, description={\ldots}}
%\newglossaryentry{Annuity}{name={Annuity}, description={\ldots}}
\newglossaryentry{Compounding}{name={Compounding}, description={\ldots}}
\newglossaryentry{Annualized_rate}{name={Annualized rate}, description={\ldots}}
\newglossaryentry{Amortized_bond}{name={Amortized bond}, description={\ldots}}
\newglossaryentry{Coupon_bond}{name={Coupon bond}, description={\ldots}}
\newglossaryentry{Zero_coupon_bond}{name={Zero coupon bond}, description={\ldots}}
\newglossaryentry{Serial}{name={Serial}, description={\ldots}}
\newglossaryentry{Annuity}{name={Annuity}, description={\ldots}}
\newglossaryentry{Bullet}{name={Bullet}, description={\ldots}}

%\newacronym{foobar}{foobar}{\ldots}

%\usepackage{marginnote}
%---- settings ----
% Comments
\newcommand{\comm}[2]{{\sf \(\spadesuit\){\bf #1: }{\rm \sf #2}\(\spadesuit\)}}
\newcommand{\mcomm}[2]{\marginpar{\scriptsize \comm{#1}{#2}}}
\newcommand{\ab}[1]{\mcomm{AB}{#1}}
\newcommand{\ja}[1]{\mcomm{JA}{#1}}

%\pagestyle{fancy}

\renewcommand{\headrulewidth}{0.5pt}
\renewcommand{\footrulewidth}{1pt}

\AtBeginEnvironment{minted}{\singlespacing%
    \fontsize{10}{10}\selectfont}

%% Font size of tables
%\let\oldtabular\tabular
%\renewcommand{\tabular}{\small\oldtabular}

\usepackage[T1]{fontenc} % font
\setlength{\parindent}{0in}
\definecolor{lightgray}{rgb}{0.9,0.9,0.9}

\newenvironment{filecode}[1][]
{\minipage{\linewidth}
\lstset{basicstyle=\ttfamily\footnotesize,frame=single,
numberstyle=\small\color{black},keywordstyle=\color{black},commentstyle=\color{black},
stringstyle=\color{black},tabsize=2,backgroundcolor=\color{lightgray},language=Haskell,#1}}
{\endminipage}
\renewcommand*\rmdefault{ppl}

\lfoot{
\begin{textblock*}{100mm}(30mm, 280mm )
\end{textblock*}
}

\pagestyle{fancy}
\fancyhf{}

\lhead{\uppercase{Hiperfit Quant Library}}
\rhead{\nouppercase{\rightmark}}

\cfoot{\thepage\ / \phantomsection\pageref*{LastPage}}

\lfoot{
\begin{textblock*}{100mm}(30mm, 280mm )
\end{textblock*}
}

%\newcolumntype{C}{>{\small}c}

%-------------------
\definecolor{listinggray}{gray}{0.98}
\definecolor{lbcolor}{rgb}{0.98,0.98,0.98}

\DeclareCaptionFont{white}{\color{white}}
\DeclareCaptionFormat{listing}{\colorbox[cmyk]{0.43, 0.35, 0.35,0.01}{\parbox{\dimexpr\textwidth-2\fboxsep\relax}{#1#2#3}}}
\captionsetup[code]{format=listing,labelfont=white,textfont=white, singlelinecheck=false, margin=0pt, font={bf,footnotesize}}


\lstnewenvironment{code}[1][]%
    {\noindent\minipage{\linewidth}%
            \lstset{#1}%
            \captionsetup{options=code}% execute options set with \captionsetup[code]{...}
    }
  {\endminipage} 

\newcommand{\fancyfootnotetext}[2]{%
  \fancypagestyle{dingens}{%
    \fancyfoot[LO,RE]{\parbox{12cm}{\footnotemark[#1]\footnotesize #2}}%
  }%
  \thispagestyle{dingens}%
}

\newcommand{\tikzmark}[1]{\tikz[overlay,remember picture] \node (#1) {};}
\newcommand{\DrawBox}[1][]{%
    \tikz[overlay,remember picture]{
    \draw[red,#1]
      ($(left)+(-0.2em,0.9em)$) rectangle
      ($(right)+(0.2em,-0.3em)$);}
}

\definecolor{comments}{rgb}{0,0.6,0}
\definecolor{strings}{rgb}{0.9,0,0}
\definecolor{keywords}{rgb}{0,0,0}
\definecolor{identifier}{rgb}{0.10,0.10,0.10}
\definecolor{background}{rgb}{0.98,0.98,0.98}


\lstdefinestyle{BashInputStyle}{
  language=haskell,
  breaklines=true,
  breakatwhitespace=true,
  frame=tblr,
  columns=fullflexible,
  backgroundcolor=\color{gray!10},
  linewidth=1.0\linewidth,
  xleftmargin=0.0\linewidth,
  xrightmargin=0.0\linewidth,
  keywordstyle= \color{keywords},
  identifierstyle=\color{identifier},
  stringstyle=\color{strings},
  commentstyle=\color{comments},
  basicstyle=\fontsize{9}{12}\color{black}\ttfamily,
  frame=single,
  framesep=0pt,
  framerule=0.0pt
}

\lstdefinestyle{Output}{
  language=haskell,
  breaklines=true,
  breakatwhitespace=true,
  frame=tblr,
  columns=fullflexible,
  backgroundcolor=\color{gray!10},
  linewidth=1.0\linewidth,
  xleftmargin=0.0\linewidth,
  xrightmargin=0.0\linewidth,
  keywordstyle= \color{keywords},
  identifierstyle=\color{identifier},
  stringstyle=\color{strings},
  commentstyle=\color{comments},
  basicstyle=\fontsize{9}{12}\color{black}\bfseries\ttfamily,
  frame=single,
  framesep=0pt,
  framerule=0.0pt
}

\lstdefinestyle{Temp}{
  breaklines=true,
  breakatwhitespace=true,
  frame=tblr,
  columns=fullflexible,
  backgroundcolor=\color{gray!10},
  linewidth=1.0\linewidth,
  xleftmargin=0.0\linewidth,
  xrightmargin=0.0\linewidth,
  keywordstyle= \color{keywords},
  identifierstyle=\color{identifier},
  stringstyle=\color{strings},
  commentstyle=\color{comments},
  basicstyle=\fontsize{9}{12}\color{black}\bfseries\ttfamily,
  frame=single,
  framesep=0pt,
  framerule=0.0pt
}

\lstset{escapechar=@,style=BashInputStyle}

\begin{document}

\title{HQL Documentation}

\title{\Title}
\author{\Author}
\date{\today}
\maketitle

\tableofcontents

\vskip 0.5\FrameSep

\newpage

%% ============================================================================================
%% Introduction
%% ============================================================================================

	\section{Introduction}
	This is the documentation of the Hiperfit Quant Library, HQL. As a reader you will
	find valuable information about the theory behind the implementation together with
	illustrative examples of real world use cases. The documentation itself is \textit{Literate Haskell}\cite{LitHaskell},
	which means you can load this file into \index{GHCi}:

	\vskip 0.5\FrameSep
	\begin{lstlisting}
	~/HQL$ ghci docs/hql.lhs
	\end{lstlisting}

%	\section{Terminology}
%	In this section you'll find the most commonly used words in this document, which are specific to the domain of finance and fixed income valuation.

%	\framebox{TODO!}


%% ============================================================================================
% Motiovation
%% ============================================================================================
        \section{Motivation}
        The motive behind HQL is to support the workflow behind analysis and pricing of fixed income instruments.
        Using a statically typed language with support for an explorative workflow using via the REPL, HQL aims to be a handy
        library and tool. Suppose we want to obtain the price of a bullet bond, or vanilla bond, with coupons and a repayment of the face value at maturity.



	\begin{figure}[h!]
	\begin{center}
	\begin{tikzpicture}[-,shorten >=1pt,auto,node distance=1.5cm,thick,minimum size=0.8cm,main node/.style={circle,draw=red,very thick}]
	\tikzstyle{selected edge} = [draw,line width=6pt,-,blue!30]

	\coordinate (belowstart) at (0,-1);
	\coordinate (start) at (0,0);
	\coordinate (stop) at (10.2,0);
	\coordinate (abovestop) at (10.2,2);
	\draw (start) -- (stop);
	\draw (start) -- (belowstart);
	\draw (stop) -- (abovestop);

	\node at (0, -1.5) () {$t=0$};

	\filldraw[draw=black, fill=lightgray]  (2,0) rectangle node {} +(0.2,0.25);
	\node at (2, 2.5) () {$t=2$};

	\filldraw[draw=black, fill=lightgray]  (4,0) rectangle node {} +(0.2,0.25);
	\node at (4, 2.5) () {$t=4$};

	\filldraw[draw=black, fill=lightgray]  (6,0) rectangle node {} +(0.2,0.25);
	\node at (6, 2.5) () {$t=6$};

	\filldraw[draw=black, fill=lightgray]  (8,0) rectangle node {} +(0.2,0.25);
	\node at (8, 2.5) () {$t=8$};

	\filldraw[draw=black, fill=blue] (10,0) rectangle node {} +(0.2,1);
	\filldraw[draw=black, fill=lightgray]  (10,1) rectangle node {} +(0.2,0.25);
	\node at (10, 2.5) () {$t=10$};

	% Legend
	\filldraw[draw=black, fill=blue] (7,-1) rectangle node {} +(0.2,0.2);
	\node at (8.3, -0.92) () {repayment};

	\filldraw[draw=black, fill=lightgray] (7,-1.5) rectangle node {} +(0.2,0.2);
	\node at (8, -1.44) () {coupon};

	\end{tikzpicture}
	\caption{Cashflow of a bullet.}
	\label{fig:bulletcf}
	\end{center}
	\end{figure}



	We'll illustrate the usage of HQL using an example where we obtain the present value of a bullet bond (seen in the figure below).
	\\\\
	The bullet bond is created through the interface with the constructor Bullet:

        \ctable[
	caption = Bullet parameters,
	pos = ht,
	doinside=\small,
	center,
	]{llll}{
	}{
	\toprule
		Paramter & Type \\
	  \midrule
          \emph{Settlement} & \texttt{Date} \\
          \emph{Maturity} & \texttt{Date} \\
          \emph{Face} & \texttt{Cash} \\
          \emph{Coupon rate} & \texttt{Double} \\
          \emph{Settlements} & \texttt{Settements} \\
          \emph{Day count convention} & \texttt{(Daycount d) => d} \\
          \emph{Business day convention} & \texttt{RollConvention} \\
	  \bottomrule
	}

	\begin{code}[caption=Pricing a Bullet]
  HQL> let settle = (read "2010-01-01")::Date
  HQL> let maturity1 = (read "2015-01-02")::Date
  HQL> let bullet  = Bullet settle maturity (Cash 100 USD) 0.1 1 Following
	\end{code}
		\vskip 0.5\FrameSep
	Next, we'll calculate the present value of the previously defined bullet object. First we need a term structure. For more information about term structures in HQL see section xx below. The term structure below is a flat one, with a fixed rate of 5.0\%.
		\vskip 0.5\FrameSep
	\begin{lstlisting}
	HQL> let ts  = AnalyticalTermStructure (x = 5.0)	
	\end{lstlisting}

	\begin{lstlisting}	
	HQL> pv bullet ts
	\end{lstlisting}
	
	yields the following output:
	\vskip 0.5\FrameSep
	\begin{lstlisting}[style=Output]
	100.00
	\end{lstlisting}
	
	\framebox{TODO!}
	
	

%% ============================================================================================
%% Interest rates
%% ============================================================================================

	\section{Interest rates and compounding}
	It seems natural to start by covering compounding and interest rates, when they together make up the fundamentsls of the concept: time value of money. They are part of the fixed income module covered in this documentation.
	Interest rates determine the rate at which money is borrowed, and exist for various maturities. Typically interest rates are quoted on an annual basis, which means the 	interest rate specifies the annual growth on capital.

\subsection{Compounding}
	Compounding refers to the future value of a nominal amount. In other words money is worth more tomorrow due to interest. The oppostite of compounding is discounting.
	When one talks about compounding, the interest rate or rate determines the rate at which the deposited amount of money will increase as a function of time.
	Compounding can be done in discrete or continous compounding fashion.


	\subsubsection{Simple compounding}
	$FV$, the future value in $T$ periods of a nominal $N$, is expressed as:

	\[
	FV_T = N(1+R_TT)
	\]

	where $R_T$ is the interest rate.

	\subsubsection{Exponential compounding}

	Future value of a nominal cash flow $N$ under exponential compounding over a period of
	$T$ years:

	\[ FV_T = N(1+R_T)^T \]

	%The present value of a nominal future cash flow $N$ is therefore

	We can create a compound factor in HQL using  exponential compounding as follows:

	\vskip 0.5\FrameSep
	\begin{lstlisting}
	HQL> let cf = compoundFactor (ExponentialRate 8.5 SemiAnnually) 2
	\end{lstlisting}
	\vskip 0.5\FrameSep

	\[ PV_T = \frac{N}{(1+R_T)^n} \]

	Periodic compounding where $n$ is the number of times interest compounded per period

	\[
	FV_T=N\left( 1 + \frac{R}{n}  \right)^{nT}
	\]

	\subsubsection{Continuous compounding}
	Continuous compounding is defined as taking the limit of the discrete compounding as n goes to infinity,
	\[
	FV=\lim_{n \to \infty} N \left(1+\frac{R}{n}\right)^{nT}=Ne^{RT}
	\]

	\ctable[
	caption = Supported frequencies
	pos = !ht,
	doinside=\small
	]{llll}{
	}{
	  \toprule
		HQL & Meaning & Frequency &  $\delta$ \\
	  \midrule
		\ttfamily{Annually} & Annually compounding & 1 & 1.0\\
		\ttfamily{SemiAnnually} & SemiAnnually compounding & 2 & 0.5 \\
		\ttfamily{Quarterly} & Quarterly compounding & 4 & 0.25\\
		\ttfamily{Monthly} & Monthly compounding & 12 & 1/12\\
		\ttfamily{Daily} & Daily compounding & 365 & 1/365\\
		\ttfamily{Other} & Other compounding & n & 1/n\\
	  \bottomrule
	}


	\ctable[
	caption = Supported compounding methods,
	pos = !ht,
	doinside=\small
	]{ll}{
	}{
	  \toprule
		HQL & Meaning \\
	  \midrule
		\ttfamily{ContinuousRate} & Continuous compounding\\
		\ttfamily{ExponentialRate} & Exponential compounding\\
		\ttfamily{SimpleRate} & Simple compounding\\
	  \bottomrule
	}

	Figure ~\ref{fig:comp01} illustrates the effect of various interest during a period of ten years.

	\begin{minipage}{\linewidth}
	\makebox[\linewidth]{%
	  \includegraphics[keepaspectratio=true,scale=1.0]{gnuplot/comp02.eps}}
	\captionof{figure}{The three different compounding methods}
	\label{fig:comp01}
	\end{minipage}

	\subsection{Example 1 - The compounding methods}
	\vskip 0.5\FrameSep
	\begin{lstlisting}
	ContinuousRate Rate deriving (Show)
	\end{lstlisting}
	\vskip 0.5\FrameSep
	\begin{lstlisting}
	SimpleRate Rate deriving (Show)
	\end{lstlisting}
	\vskip 0.5\FrameSep
	\begin{lstlisting}
	ExponentialRate Rate Frequency
	\end{lstlisting}

	Figure ~\ref{fig:comp01} illustrates the effect of various interest during a period of ten years.

	\begin{minipage}{\linewidth}
	\makebox[\linewidth]{%
	  \includegraphics[keepaspectratio=true,scale=1.0]{gnuplot/comp01.eps}}
	\captionof{figure}{Continuous compounding with various rates}
	\label{fig:comp01}
	\end{minipage}

	\subsection{Example 2 - The effect of the interest rate}

	Let's look at an example how to calculate the compound factor using an interest rate of 10\% during 10 years:

	\vskip 0.5\FrameSep
	\begin{lstlisting}
	compoundFactor (ExponentialRate 10 Annually) 10	
	\end{lstlisting}
	\vskip 0.5\FrameSep

	yields the following output:
	\vskip 0.5\FrameSep
	\begin{lstlisting}[style=Output]
	2.714080846608224
	\end{lstlisting}

	It's rather easy to make HQL produce a list of compound factors using a list of years:

	\vskip 0.5\FrameSep
	\begin{lstlisting}
	map (\r -> compoundFactor (ExponentialRate r Annually) 10) [3,5,10]
	\end{lstlisting}
	\vskip 0.5\FrameSep

	yields the following output:
	\vskip 0.5\FrameSep
	\begin{lstlisting}[style=Output]
	[1.3439163793441222, 1.6288946267774422, 2.5937424601000023]
	\end{lstlisting}

	\subsection{Example 3 - Compounding frequencies}

	The table below illustrates the effect of compounding at various frequencies. The nominal rate
	is the annualized interest rate, compounded once per year. As the frequency increases, the discrete
	rate almost equals the continous one.

	\ctable[
	caption = Interest rates at various compounding frequencies,
	pos = ht,
	doinside=\small
	]{lllllll}{
	}{
	\toprule
	 Nominal Rate &  Semi-Annual & Quarterly &  Monthly &  Weekly &  Daily  & Continuous \\
	  \midrule
		1.0000       & 1.0025       & 1.0038     & 1.0046   & 1.0049  & 1.0050  & 1.0050      \\
		2.0000       & 2.0100       & 2.0151     & 2.0184   & 2.0197  & 2.0201  & 2.0201      \\
		5.0000       & 5.0625       & 5.0945     & 5.1162   & 5.1246  & 5.1267  & 5.1271      \\
		10.0000      & 10.2500      & 10.3813    & 10.4713  & 10.5065 & 10.5156 & 10.5171     \\
		15.0000      & 15.5625      & 15.8650    & 16.0755  & 16.1583 & 16.1798 & 16.1834     \\
		20.0000      & 21.0000      & 21.5506    & 21.9391  & 22.0934 & 22.1336 & 22.1403     \\
		25.0000      & 26.5625      & 27.4429    & 28.0732  & 28.3256 & 28.3916 & 28.4025     \\
		50.0000      & 56.2500      & 60.1807    & 63.2094  & 64.4788 & 64.8157 & 64.8721     \\
	  \bottomrule
	}

	In this example the table below will be generated by HQL.

	\vskip 0.5\FrameSep
	\begin{lstlisting}
	continuousRate $ ExponentialRate 1.0000 Annually
	\end{lstlisting}
	\vskip 0.5\FrameSep
	\begin{lstlisting}[style=Output]
	2.714080846608224
	\end{lstlisting}

	Instead, using a semi annual compounding with a value from table 3:

	\begin{lstlisting}
	continuousRate $ ExponentialRate 1.0025 SemiAnnually
	\end{lstlisting}
	\vskip 0.5\FrameSep
	\begin{lstlisting}[style=Output]
	2.714080846608224
	\end{lstlisting}

	Instead, using a quarterly compounding with a value from table 3:

	\begin{lstlisting}
	continuousRate $ ExponentialRate 1.0025 Quarterly
	\end{lstlisting}
	\vskip 0.5\FrameSep
	\begin{lstlisting}[style=Output]
	2.714080846608224
	\end{lstlisting}


%% ============================================================================================
%% Compounding
%% ============================================================================================
\begin{comment}
	\section{Compounding}
	Compounding refers to the future value of a nominal amount. In other words money is worth more tomorrow due to interest and inflation. The effect of inflation is omitted in HQL. The oppostite of compounding is discounting.
	When one talks about compounding, the interest rate or rate determines the rate at which the deposited amount of money will increase as a function of time.
	Compounding is often separated into discrete and continous compounding.


	\subsubsection{Linear compounding}
	$FV$, the future value in $T$ periods of a nominal $N$, is expressed as:

	\[
	FV_T = N(1+R_TT)
	\]

	where $R_T$ is the interest rate.

	\subsubsection{Discrete compounding}
	Discrete compounding refers to compounding with fixed and known intervals of compounding, typically annually, semi-annually or quarterly.

	\vskip 0.5\FrameSep
	\begin{lstlisting}
	HQL> let TODO
	\end{lstlisting}
	\vskip 0.5\FrameSep



	\subsubsection{Exponential compounding}

	Future value of a nominal cash flow $N$ under exponential compounding over
	$n$ periods:

	\[ FV_T = N(1+R_T)^n \]

	%The present value of a nominal future cash flow $N$ is therefore

	We can create a compound factor in HQL using  exponential compounding as follows:

	\vskip 0.5\FrameSep
	\begin{lstlisting}
	HQL> let TODO
	\end{lstlisting}
	\vskip 0.5\FrameSep

	\[ PV_T = \frac{N}{(1+R_T)^n} \]

	Periodic compounding where $n$ is the number of times interest compounded per period

	\[
	FV_T=N\left( 1 + \frac{R}{n}  \right)^{nT}
	\]

	\subsection{Continuous compounding}
	Continuous compounding is defined as taking the limit of the discrete compounding as n goes to infinity,
	\[
	FV=\lim_{n \to \infty} N \left(1+\frac{R}{n}\right)^{nT}=Ne^{RT}
	\]

	The continous compounding factors in HQL is defined as follows:

	Example:

	\subsection{Calculating present and future value}
	%To calculate the present value of a future cash flow we use the discount factor
	%to scale the amount:

	%We can calculate the future value by using the inverse of the discount factor above:

	To calculate present and future value in HQL, two objects are of particular interest:
	the DiscountFactor and the CompoundFactor. Let's start by using the DiscountFactor in an example
	where we have a monthly compounded interest rate of 8.5\% during a period of 3 years:
	\vskip 0.5\FrameSep
	\begin{lstlisting}
	HQL> let let df = discountFactor (InterestRate (Periodic 12) 8.5) 3.0 0.0
	\end{lstlisting}
	We can use the previously defined DiscountFactor to discount an arbitrary amount. Let's say
	\$1000.00:
	\vskip 0.5\FrameSep
	\begin{lstlisting}
	HQL> let df*1000.0
	\end{lstlisting}
	yields the following output:
	\vskip 0.5\FrameSep
	\begin{lstlisting}[style=Output]
	775.6133702070988
	\end{lstlisting}
\end{comment}

%% ============================================================================================
%% Discounting
%% ============================================================================================

	\section{Discounting}
	Discounting is the present value of a future cash flow, threfore discounting and compounding and are each others inverse. This means
	the discounting of a compounding with the same interest rate and time to maturity will be equal to the initial amount.

	\subsection{Discount factors}
	Discount factors in HQL are dealt with in the same way as compound factors, they take interest rates and a time offset as arguments.

	\subsection{The exponential discount factor}

	\begin{lstlisting}
	HQL> let discountFactor (ExponentialRate 8.5 SemiAnnually) 2
	\end{lstlisting}

	\subsection{The continuous discount factor}
	The \textit{continuous discount factor} with maturity $T$ at time $t$ is defined as,
	\[
	p(t,T)=e^{-r(T-t)}; T>t
	\]
	Which equals the \textit{zero-coupon unit bond}, with face $N=1$. The discount factor
	can be seen as entities in a vector, the \textit{discount function},

\begin{comment}
	\vskip 0.5\FrameSep
	\begin{lstlisting}
	HQL> let zipWith (*) (discountFactors (interestRate Continuous 5.0) 5 4 <*> pure 0) [1000,1500,1000,2000]
	\end{lstlisting}
	\vskip 0.5\FrameSep
	yields the following output:
	\vskip 0.5\FrameSep
	\begin{lstlisting}[style=Output]
	[951.229424500714,1357.2561270539393,860.7079764250578,1637.4615061559637]
	\end{lstlisting}
	\vskip 0.5\FrameSep
\end{comment}
	\[
	\mathbf{df} = (p(t,T_1), ..., p(t,T_n))
	\]

	In the same way, cash flows can be considered entities in a vector \cite{williamfsharp},

	\[
	\mathbf{cf} = (c_1, ..., c_i)
	\]

	For example, if we were to calculate the present value of a series of future cash flows,
	\[
	\mathbf{cf} = (\$1000.00,\$1500.00,\$1000.00,\$2000.00)
	\]
	We can multiply by the discount function with $T_n=n$ years and the rate at 5.00\% continuously compounded,
	\[
	p(0) = \mathbf{df}\cdot\mathbf{cf} = (e^{-rT_1},...,e^{-rT_n})\cdot(p(0,T_1), ...,p(0,T_4))=
	\]
	\[
	(e^{-0.05\cdot1},e^{-0.05\cdot2},e^{-0.05\cdot3},e^{-0.05\cdot4}) \cdot (\$1000.00,\$1500.00,\$1000.00,\$2000.00)=
	\]
	\[
	(\$951.23,\$1357.26,\$860.71,\$1637.46)
	\]

	The above calculation is done in HQL using the function \textit{zipWith} taking the discount function and the list of future cash flows as arguments.
	\vskip 0.5\FrameSep
	\begin{lstlisting}
	HQL> let zipWith (*) (discountFactors (interestRate Continuous 5.0) 5 4 <*> pure 0) [1000,1500,1000,2000]
	\end{lstlisting}
	\vskip 0.5\FrameSep
	yields the following output:
	\vskip 0.5\FrameSep
	\begin{lstlisting}[style=Output]
	[951.229424500714,1357.2561270539393,860.7079764250578,1637.4615061559637]
	\end{lstlisting}
	\vskip 0.5\FrameSep
	A slightly modified version will return the sum of the cash flows. This is the basis of bond pricing, which is covered in section \nameref{sec:fi}.
        In the example below, the function \texttt{<*>} is used to initialize the discount factors with an offset of 0.
	\vskip 0.5\FrameSep
	\begin{lstlisting}
	HQL> let sum $ zipWith (*) (discountFactors (interestRate Continuous 5.0) 5 4 <*> pure 0) [1000,1500,1000,2000]
	\end{lstlisting}
	\vskip 0.5\FrameSep
	yields the following output:
	\vskip 0.5\FrameSep
	\begin{lstlisting}[style=Output]
	4806.655034135675
	\end{lstlisting}

%% ============================================================================================
%% More about interest rates
%% ============================================================================================
	\section{More about interest rates}
	Interest rate instruments refer to traded and listed interest rates, such as LIBOR spot and LIBOR forward rates.
	Using the interface of HQL, it's possible to create arbitrary spot and forward rates. LIBOR is used here for illustrative purposes,
	and because HQL provides an interface to work with LIBOR rates.

	\ctable[
	caption = Typical market data
	pos = ht,
	doinside=\small,
	center,
	]{llll}{
	}{
	\toprule
		Maturity (months) & Spot rate & Forward & Rate \\
	  \midrule
		3 & 4.5\% & $F_{0,3}$ & 4.5\% \\
		6 & 4.3\% & $F_{3,3}$ & 4.05\% \\
		9 & 4.2\% & $F_{6,3}$ & 3.92\% \\
		12 & 4.0\% & $F_{9,3}$ & 3.3\% \\
	  \bottomrule
	}

	\subsection{Spot rates}
	Spot rates refers to rates that are observable at present time in the market.
	Spot rates are quoted on an annualized basis. Spot rates are for example LIBOR spot rates, see below.

	\subsection{Forward rates}
	\framebox{TODO: Add figure to illustrate the concept}
	Forward rates are used to price future cash flows and bonds, and to construct term structures. Forward rates indicate the interest rate between two future dates [S,T]
	\[
	(1+R_T)^T=(1+R_S)^S(1+R(t;S,T))^{T-S}; T>S
	\]
	Solving for the forward rate yields
	\[
	R(t=0,S,T)=\left( \frac{(1+R_T)^T}{(1+R_S)^S} - 1 \right)^{\frac{1}{T-S}};T>S
	\]

	Continuous compounding version of the above

	\[
	e^{R_TT}=e^{R_SS}e^{R(t;S,T)(T-S)};T>S
	\]
	Solving for the forward rate
	\[
	R(t;S,T)=\frac{R_TT-R_SS}{T-S}
	\]

	Let's look at an example of how to determine the forward rate given two time periods and interest rates. The input parameters are listed in the table below.

	\ctable[
	caption = Input parameters to example,
	pos = !ht,
	width=80mm,
	center,
	doinside=\small
	]{ll}{
	}{
	  \toprule
		Parameter & Value \\
	  \midrule
		Compounding & Semi-annually ($n=2$)\\
		$\tau_1$ & 0.25\\
		$\tau_2$ & 1.5 \\
		$R_1$ & 6.65 \\
		$R_2$ & 7.77 \\
	  \bottomrule
	}

	\vskip 0.5\FrameSep
	\begin{lstlisting}
	HQL> let (((discountFactor (InterestRate (Periodic 2) 6.65) 0.25 0.0)/(discountFactor (InterestRate (Periodic 2) 7.77) 1.5))**(1/(2*1.25))-1)*2*100
	\end{lstlisting}
	yields the following output:
	\vskip 0.5\FrameSep
	\begin{lstlisting}[style=Output]
	7.994727369824384
	\end{lstlisting}

	\subsection{LIBOR}
        The LIBOR is choosen as an example of a quoted rate because it's the most widespread reference rate today.
        The LIBOR (London InterBank Offered Rate) interest rate is the most important interbank rate
	used for fixed income valuation. LIBOR rates are quoted on a simple compounding basis with maturities
	from overnight to 12 months.

	The rate quoted is the annualized rate, to calculate the actual rate we do as follows.

        (Insert formula and HQL example)

	\begin{minipage}{\linewidth}
	\makebox[\linewidth]{%
	  \includegraphics[keepaspectratio=true,scale=1.0]{gnuplot/libor3.eps}}
	\captionof{figure}{Historical LIBOR rates USD}
	\label{fig:comp01}
	\end{minipage}


	LIBOR comes in 8 maturities
	5 different currencies
	\\
	\\
	Table of LIBOR rates per 2013-12-26,

	\ctable[
	caption = LIBOR rates 2013-12-26,
	pos = ht,
	width=60mm,
	center,
	doinside=\small
	]{ll}{
	}{
	  \toprule
		Period & Annual rate \\
	  \midrule
		1 month & 0.17\% \\
		3 month & 0.24\% \\
		6 month & 0.35\% \\
		12 month & 0.59\% \\
	  \bottomrule
	}


	%$R$ is the annual rate
	%The value after m compounding periods with n compounding periods per year

        The discount factor $p(t,S)$, is also called a nominal zero bond, is the fair value at time $t$ with maturity $S$.

	\[
	1+(T-S)L=\frac{p(t,S)}{p(t,T)}
	\]
	\[
	e^{r(T-S)}=\frac{p(t,S)}{p(t,T)}
	\]
	\\
	\textbf{LIBOR forward rate}
	Simply-compounded forward rate $[S,T]$ prevailing at time $t$,
	\[
	L(S,T) = -\frac{p(S,T)-p(t,S)}{(T-S)p(t,T)}
	\]
	\\
	\textbf{LIBOR spot rate},

	\[
	L(S,T)=\frac{p(S,T)-1}{(T-S)p(S,T)}
	\]
	\\
	\textbf{LIBOR forward rate},
	Continuously compounded forward rate $[S,T]$ prevailing at time $t$,

	\[
	R(t;S,T)=-\frac{\log{p(t,T)}-\log{p(t,S)}}{T-S}
	\]
	\\
	\textbf{LIBOR spot rate},
	\[
	R(S,T)=\frac{\log{p(S,T)}}{T-S}
	\]
	\\
	\textbf{Instantaneous forward rate},
	\[
	f(t,T)=-\frac{\partial\log{p(t,T)}}{\partial{T}}
	\]
	\\
	\textbf{Instantaneous short rate},
	\[
	r(t)=f(t,t)
	\]

	For example, the three-month forward LIBOR for the period $[S,T]$, where $T=S+\tau$ and $\tau=1/4$,
	\[
	L(t,T) = F(t;S,S+\tau) = F(t;S,T), T>S
	\]

	\subsection{Using LIBOR rates}
	LIBOR rates are quoted intra daily, but sometimes it is of interest to convert between spot rates and forward rates if one or the
	other is not quoted or available.

        \subsection{A note on stochastic interest rates}
        Stochastic interest rates, or short rates, are not supported by HQL in the sense that it's used for valuation of instruments.
        This section illustrates how stochastic interest rates are implemented and used in HQL, to give an understanding and to support for future work.
        \\\\
        Below is an example of a short rate model, the Merton's model from 1973:

        \[
        r_t = r_0 + at + \sigma W^{*}_{t}
        \]
        
        Where the short rate $r_t$ is expressed as a function of $t$ and the one-dimensional Brownian motion $W^{*}_{t}$.

	\section{The term structure}

	A \textbf{term structure of interest rates} refers to the general relation between yield and maturity that exists in a given bond market. If the term structure is derived from zero-coupon bonds it forms a good basis for pricing analysis of \emph{any} bond.\\

	A \textbf{yield curve} is a plot of a specific set of bond yields as a function of their maturity. This is not good for general pricing or valuation purposes, except for the specific bonds plotted. The key difference between a term structure and a yield curve is that we discount \emph{any} cashflow by using former, while the latter is only applicable for bonds who's yields correspond. Figure SOMETHING

	The term structure or the yield curve.
	Created from zero coupon bonds, zero yield curve
	\[
	y(t,T)=-\frac{\log{p(t,T)}}{T-t}
	\]
	The yield curve is constructed by fixating $t$,
	\[
	T\to y(t,T)
	\]
	To obtain the discount factor from a yield curve at time $t$,
	\[
	e^{y(t,T)(T-t)}=e^{-\log{p(t,T)}}\iff e^{y(t,T)(T-t)}=\frac{1}{p(t,T)}
	\]
	\[
	\iff p(t,T)=e^{-y(t,T)(T-t)}
	\]
	We can price a bond from the yield curve, given face value $N$,
	\[
	p(t,T)=e^{-y(t,T)(T-t)}\cdot N
	\]


	\begin{figure}[h!]
	\begin{center}
	\begin{tikzpicture}[-,shorten >=1pt,auto,node distance=1.5cm,thick,minimum size=0.8cm,main node/.style={circle,draw=red,very thick}]
	\tikzstyle{selected edge} = [draw,line width=6pt,-,blue!30]

	\coordinate (belowstart) at (0,-1);
	\coordinate (xaxis) at (10,0);
	\coordinate (origo) at (0,0);
	\coordinate (yaxis) at (0,5);

	% Draw axes
	\draw[->] (origo) -- (xaxis) node[right] {Time to maturity};
	\draw[->] (origo) -- (yaxis) node[above] {Interest rate};

	% Flat TS
	\draw[thick,draw=blue] (0,1.5) -- (10,1.5);
	% "Normal" TS
	\draw[thick,draw=green] (0,0) parabola[bend at end] (10,5);
	% Inverse TS
	\draw[thick,draw=red] (0,5) parabola[bend at end] (10,0.5);

	% Legend
	\filldraw[draw=black, fill=blue] (5.2,-1) rectangle node {} +(0.2,0.2);
	\node at (6, -0.92) () {Flat};

	\filldraw[draw=black, fill=green] (0,-1) rectangle node {} +(0.2,0.2);
	\node at (1.55, -0.92) () {Logarithmic};

	\filldraw[draw=black, fill=lightgray] (3,-1) rectangle node {} +(0.2,0.2);
	\node at (4.15, -0.92) () {Inverted};

	\end{tikzpicture}
	\caption{An example of different term structures.}
	\label{fig:anc}
	\end{center}
	\end{figure}


	\subsection{Flat term structure}
	The flat term structure is nothing more than a flat yield curve, with a fixed yield for all maturities.

	\begin{lstlisting}
	HQL> let let flatTerm x = 5.0
	\end{lstlisting}

	\subsection{Analytic term structure}
	The analytical term structure describes the yield as an analytical function. Consider the example below:

	\begin{lstlisting}
	HQL> let let AnalyticTerm x = 5 + (1/2)*sqrt(x)
	\end{lstlisting}

	\subsection{Interpolated term structure}
	The interpolated term structure is not supperted yet by HQL.

	\subsubsection{Example - Treasury term structure}
	In this example we are going to create a term structure from the data available for
	US treasury bonds. Below is a table with data from 2013-12-26.

	\ctable[
	caption = US Treasury data 2013-12-26,
	pos = ht,
	width=80mm,
	center,
	doinside=\small
	]{ll}{
	}{
	  \toprule
		Instrument & Yield \\
	  \midrule
		3-month treasury bill & 0.07\% \\
		1-year treasury bill & 0.13\% \\
		2-year treasury note & 0.42\% \\
		5-year treasury note & 1.74\% \\
		10-year treasury note & 3\% \\
		30-year treasury bond & 3.92\% \\
	  \bottomrule
	}



%% ============================================================================================
%% Bonds and cash flows
%% ============================================================================================

	\section{Bonds and cash flows}
	Bonds refers to one or more cash flows to be received in the future, and are thus dept instruments. One typically talk about the issuer and the holder of a bond. The issuer receives the intial payment of the bond, and the hold will receive one or more payments according to the predefined cash flows in the contract.

	(Several details about a bond is of interest: face, maturity, coupon rate, issue date etc...)

	\label{sec:fi}

	\subsection{Zero coupon bond}
	The simples form of a bond is the zero coupon bond. This bond has only one cash flow specified, which till take place at the maturity date of the bond. Zero coupon bonds are sometimes refered to as discount bonds, because they are in fact discount factors scaled by the face value N.

	\vskip 0.5\FrameSep
	\begin{lstlisting}
	Zero settle maturity (Cash 100 SEK) rate1 Preceding ModifiedFollowing
	\end{lstlisting}
	\vskip 0.5\FrameSep


	\subsection{Fixed coupon bond}
	Series of payments called coupons and a nominal value N (face value)
	A number of future payment dates $T_1 < ... < T_n$ where is the maturity of the bond
	A sequence of deterministic coupons $c_1, ..., c_n$

	\[ c_{T_i} = \left\{
	\begin{array}{ll}
	  Nc, & \text{for } i=1,...,n-1,  \\
	  N(c+1), &\text{for } i=n.
	\end{array} \right.\]

	\[
	p(t) = N \cdot p(t,T_n)+\sum_{i=1}^{n} (c_ip(t,T_i))
	\]
	From above, we can see that the fixed coupon bond can be reconstructed by a portfolio of zero
	bonds (scaled by the coupon), $p(t,T)$ is a zero bond with N=1.
	\[
	p(t)=\left[p(t,T_n)+r\delta\sum_{i=1}^{n} p(t,T_i)\right]\cdot N
	\]
	Where $r$ is the coupon rate. For a standardized coupon bond the time intervals will be equally
	spaced according to
	\[
	T_i=T_0+i\delta
	\]

	Convert above to a list of payments at offset times. Where denotes the time interval in years
	according to the day-count convention, see below.

	\subsection{Example - Calculating present value}

	\ctable[
	caption = Input parameters to example,
	pos = h,
	center,
	doinside=\small
	]{llllll}{
	}{
	  \toprule
		Maturity (years) & 1 & 2 & 3 & 4 & 5\\
	  \midrule
		Interest rate & 3.00\% & 3.25\% & 3.50\% & 3.55\% & 3.30\%\\
	  \bottomrule
	}

	Let's consider a bond maturing in 5 years, paying a coupon of 10\% and a face of 1000,
	\ctable[
	caption = Bond price,
	pos = h,
	center,
	doinside=\small
	]{lllll}{
	}{
	  \toprule
		Maturity & Interest rate & Cash flow & Present value \\
	  \midrule
		1 & 3.00\% & 100 & 97.09 \\
		2 & 3.25\% & 100 & 93.80 \\
		3 & 3.50\% & 100 & 90.19 \\
		4 & 3.55\% & 100 & 86.98 \\
		5 & 3.30\% & 1100 & 935.17 \\
	\midrule
		Price bond  &     &  & \textbf{1303.23} \\
	  \bottomrule
	}

	\framebox{TODO: Adapt HQL to handle this case}

	\subsection{Floating coupon bond}
	Not yet supported by the HQL interface. For an example how HQL can be used to estimate stochastic interest rates, see section Y.

	%% ---------------------------------------------------------------
	%% Yield to maturity
	%% ---------------------------------------------------------------
	\subsection{Yield to maturity}

	The yield of a bond is the discount rate that makes the future value of all future
	cashflows equal to the present value. For coupon payments $C_i$, bond price $N_t^T$
	at time $t$ with maturity $T$, the yield $y_t$ satisfies:

	\[
	N_t^T = \sum_{T_i > t} C_i (1+y_t)^{-(T_i - t)}, \hspace{2cm} \forall\; T_i < T
	\]

	For zero-coupon bonds, we therefore have the following for a maturity $T$ at
	time $t$:

	\[
	N_t^T = (1+y_t^T)^{-(T-t)}
	\]

	with yield, $y_t^T$:

	\[
	y_t^T = \sqrt[T-t]{N_t^T} - 1
	\]

	\ab{Note on perpetual bonds}

	%\subsection{Duration and convexity}

	\section{Standard bond types}

	We now present the bond types supported in HQL. The bond types are found the module Instruments.FixedIncome.Bonds.



	%% ---------------------------------------------------------------
	%% Annuity
	%% ---------------------------------------------------------------
	\subsection{Annuity}

	An annuity is an example of an amortized bond meaning that the debitor repays the
	face value over its lifetime. Payments are distributed equally over all the settlement
	dates, increasing and decreasing the repayment amount and coupon respectively (see figure
	\ref{fig:annuitycf}.

        \ctable[
	caption = Annuity parameters,
	pos = ht,
	doinside=\small,
	center,
	]{llll}{
	}{
	\toprule
		Paramter & Type \\
	  \midrule
          Settlement & \texttt{Date} \\
          Maturity & \texttt{Date} \\
          Face & \texttt{Cash} \\
          Coupon rate & \texttt{Double} \\
          Settlements & \texttt{Settements} \\
          Day count convention & \texttt{(Daycount d) => d} \\
          Business day convention & \texttt{RollConvention} \\
	  \bottomrule
	}

	\begin{figure}[h!]
	\begin{center}
	\begin{tikzpicture}[-,shorten >=1pt,auto,node distance=1.5cm,thick,minimum size=0.8cm,main node/.style={circle,draw=red,very thick}]
	\tikzstyle{selected edge} = [draw,line width=6pt,-,blue!30]

	\coordinate (belowstart) at (0,-1);
	\coordinate (start) at (0,0);
	\coordinate (stop) at (10.2,0);
	\coordinate (abovestop) at (10.2,2);
	\draw (start) -- (stop);
	\draw (start) -- (belowstart);
	\draw (stop) -- (abovestop);

	\node at (0, -1.5) () {$t=0$};

	\filldraw[draw=black, fill=blue] (2,0) rectangle node {} +(0.2,1.5);
	\filldraw[draw=black, fill=lightgray]  (2,1.5) rectangle node {} +(0.2,0.25);
	\node at (2, 2.5) () {$t=2$};

	\filldraw[draw=black, fill=blue] (4,0) rectangle node {} +(0.2,1.3);
	\filldraw[draw=black, fill=lightgray]  (4,1.3) rectangle node {} +(0.2,0.45);
	\node at (4, 2.5) () {$t=4$};

	\filldraw[draw=black, fill=blue] (6,0) rectangle node {} +(0.2,1.1);
	\filldraw[draw=black, fill=lightgray]  (6,1.1) rectangle node {} +(0.2,0.65);
	\node at (6, 2.5) () {$t=6$};

	\filldraw[draw=black, fill=blue] (8,0) rectangle node {} +(0.2,1);
	\filldraw[draw=black, fill=lightgray]  (8,1) rectangle node {} +(0.2,0.75);
	\node at (8, 2.5) () {$t=8$};

	\filldraw[draw=black, fill=blue] (10,0) rectangle node {} +(0.2,0.85);
	\filldraw[draw=black, fill=lightgray]  (10,0.85) rectangle node {} +(0.2,0.9);
	\node at (10, 2.5) () {$t=10$};

	% Legend
	\filldraw[draw=black, fill=blue] (7,-1) rectangle node {} +(0.2,0.2);
	\node at (8.3, -0.92) () {repayment};

	\filldraw[draw=black, fill=lightgray] (7,-1.5) rectangle node {} +(0.2,0.2);
	\node at (8, -1.44) () {coupon};

	\end{tikzpicture}
	\caption{Cashflow of an annuity.}
	\label{fig:annuitycf}
	\end{center}
	\end{figure}

	\subsubsection{Example}
	Suppose we want to price an annuity maturing in 5 years, a face of \$1000.00 at an annual interest rate of 5.00\%.
	This can be done in continuous time integrating the continuous discount factor $e^-rt$:
	\[
	p(0)=N\cdot \int_{t=0}^T \mathrm{e}^{-rt}\,\mathrm{d}t=\frac{N}{R}\cdot (e^{-rt}-e^{-rT})=\frac{N}{R}(1-e^{-rT})
	\]
	Plugging in the numbers from above gives:
	\[
	p(0)=\frac{\$1000.00}{0.05}(1-e^{-\log(1+0.005)\cdot 5})\approx \$4329.48
	\]
	\begin{lstlisting}
	Annuity settle maturity2 (Cash 100 GBP) rate2 stms2 ACTACT ModifiedFollowing
	\end{lstlisting}
	yields the following output:
	\vskip 0.5\FrameSep
	\begin{lstlisting}[style=Output]
	775.6133702070988
	\end{lstlisting}

	\subsection{Bullet}
	Bullet has a fixed rate coupon which is paid at every settlement. No repayments before maturity. Figure \ref{fig:bulletcf} shows the cashflows.

	\[ p(t,T) = \sum_{t=1}^{T}\frac{C}{(1+r)^t} + \frac{N}{(1+r)^T} \]

	\begin{figure}[h!]
	\begin{center}
	\begin{tikzpicture}[-,shorten >=1pt,auto,node distance=1.5cm,thick,minimum size=0.8cm,main node/.style={circle,draw=red,very thick}]
	\tikzstyle{selected edge} = [draw,line width=6pt,-,blue!30]

	\coordinate (belowstart) at (0,-1);
	\coordinate (start) at (0,0);
	\coordinate (stop) at (10.2,0);
	\coordinate (abovestop) at (10.2,2);
	\draw (start) -- (stop);
	\draw (start) -- (belowstart);
	\draw (stop) -- (abovestop);

	\node at (0, -1.5) () {$t=0$};

	\filldraw[draw=black, fill=lightgray]  (2,0) rectangle node {} +(0.2,0.25);
	\node at (2, 2.5) () {$t=2$};

	\filldraw[draw=black, fill=lightgray]  (4,0) rectangle node {} +(0.2,0.25);
	\node at (4, 2.5) () {$t=4$};

	\filldraw[draw=black, fill=lightgray]  (6,0) rectangle node {} +(0.2,0.25);
	\node at (6, 2.5) () {$t=6$};

	\filldraw[draw=black, fill=lightgray]  (8,0) rectangle node {} +(0.2,0.25);
	\node at (8, 2.5) () {$t=8$};

	\filldraw[draw=black, fill=blue] (10,0) rectangle node {} +(0.2,1);
	\filldraw[draw=black, fill=lightgray]  (10,1) rectangle node {} +(0.2,0.25);
	\node at (10, 2.5) () {$t=10$};

	% Legend
	\filldraw[draw=black, fill=blue] (7,-1) rectangle node {} +(0.2,0.2);
	\node at (8.3, -0.92) () {repayment};

	\filldraw[draw=black, fill=lightgray] (7,-1.5) rectangle node {} +(0.2,0.2);
	\node at (8, -1.44) () {coupon};

	\end{tikzpicture}
	\caption{Cashflow of a bullet.}
	\label{fig:bulletcf}
	\end{center}
	\end{figure}

\ctable[
	caption = Bullet parameters,
	pos = ht,
	doinside=\small,
	center,
	]{llll}{
	}{
	\toprule
		Paramter & Type \\
	  \midrule
          Settlement & \texttt{Date} \\
          Maturity & \texttt{Date} \\
          Face & \texttt{Cash} \\
          Coupon rate & \texttt{Double} \\
          Settlements & \texttt{Settements} \\
          Day count convention & \texttt{(Daycount d) => d} \\
          Business day convention & \texttt{RollConvention} \\
	  \bottomrule
	}
	\begin{lstlisting}
	HQL> let settle = (read "2010-01-01")::Date
	HQL> let maturity1 = (read "2015-01-02")::Date
	HQL> let bullet  = Bullet settle maturity (Cash 100 USD) 0.1 1 Following
	\end{lstlisting}


	\subsection{Consol}
	Consol has a fixed rate coupon and never terminates and there is no payments at any
	settlement, only interest (the coupon) is paid. Figure \ref{fig:consolcf} shows this
	pictorially.

	\[
	p(t) = N \cdot p(t,T_n)+\sum_{i=1}^{n} (c_i \cdot p(t,T_i))
	\]
	\[
	\lim_{x \to \infty} p(t,\infty)=\left[r\delta\sum_{i=1}^{\infty} p(t,T_i)\right]\cdot N =
	\sum_{i=0}^{\infty} p(t,T_i) \cdot \frac{\delta N}{e^{rT}} \cdot \to \frac{\delta N}{e^{rT}}\left[ \frac{e^{rT}}{r}\right] = \frac{\delta N}{r}
	\]
	Which means the present value of the consol bond is represented by the the following formula:
	\[
	p(t) = \frac{\delta N}{r}
	\]

	\framebox{p(t) function of time?}

	\begin{figure}[h!]
	\begin{center}
	\begin{tikzpicture}[-,shorten >=1pt,auto,node distance=1.5cm,thick,minimum size=0.8cm,main node/.style={circle,draw=red,very thick}]
	\tikzstyle{selected edge} = [draw,line width=6pt,-,blue!30]

	\coordinate (belowstart) at (0,-1);
	\coordinate (start) at (0,0);
	\coordinate (stop) at (9.2,0);
	\coordinate (dotstop) at (10.2,0);
	\draw (start) -- (stop);
	\draw (start) -- (belowstart);
	\draw[dotted] (stop) -- (dotstop);

	\node at (0, -1.5) () {$t=0$};

	\filldraw[draw=black, fill=lightgray]  (2,0) rectangle node {} +(0.2,0.25);
	\node at (2, 2.5) () {$t=2$};

	\filldraw[draw=black, fill=lightgray]  (4,0) rectangle node {} +(0.2,0.25);
	\node at (4, 2.5) () {$t=4$};

	\filldraw[draw=black, fill=lightgray]  (6,0) rectangle node {} +(0.2,0.25);
	\node at (6, 2.5) () {$t=6$};

	\filldraw[draw=black, fill=lightgray]  (8,0) rectangle node {} +(0.2,0.25);
	\node at (8, 2.5) () {$t=8$};

	% Legend
	\filldraw[draw=black, fill=lightgray] (7,-1) rectangle node {} +(0.2,0.2);
	\node at (8.3, -0.92) () {coupon};

	\end{tikzpicture}
	\caption{Cashflow of a consol.}
	\label{fig:consolcf}
	\end{center}
	\end{figure}

\ctable[
	caption = Consol parameters,
	pos = ht,
	doinside=\small,
	center,
	]{llll}{
	}{
	\toprule
		Paramter & Type \\
	  \midrule
          Settlement & \texttt{Date} \\
          Maturity & \texttt{Date} \\
          Face & \texttt{Cash} \\
          Coupon rate & \texttt{Double} \\
          Settlements & \texttt{Settements} \\
          Day count convention & \texttt{(Daycount d) => d} \\
          Business day convention & \texttt{RollConvention} \\
	  \bottomrule
	}
	\begin{lstlisting}
	HQL> let settle = (read "2010-01-01")::Date
	HQL> let maturity1 = (read "2015-01-02")::Date
	HQL> let bullet  = Consol settle maturity (Cash 100 USD) 0.1 1 Following
	\end{lstlisting}

	\subsection{Serial}

	Serial is an amortized bond with a fixed rate coupon, the repayments are spread evenly over all remaining settlements and the coupon payments, $C_t$ decline over time as depicted in figure
	\ref{fig:serialcf}.

	\[ p(t,T) = \sum_{t=1}^{T}\frac{C_t}{(1+r)^t} + \frac{N}{(1+r)^T} \]

	\begin{figure}[h!]
	\begin{center}
	\begin{tikzpicture}[-,shorten >=1pt,auto,node distance=1.5cm,thick,minimum size=0.8cm,main node/.style={circle,draw=red,very thick}]
	\tikzstyle{selected edge} = [draw,line width=6pt,-,blue!30]

	\coordinate (belowstart) at (0,-1);
	\coordinate (start) at (0,0);
	\coordinate (stop) at (10.2,0);
	\coordinate (abovestop) at (10.2,2);
	\draw (start) -- (stop);
	\draw (start) -- (belowstart);
	\draw (stop) -- (abovestop);

	\node at (0, -1.5) () {$t=0$};

	\filldraw[draw=black, fill=blue] (2,0) rectangle node {} +(0.2,1);
	\filldraw[draw=black, fill=lightgray]  (2,1) rectangle node {} +(0.2,1);
	\node at (2, 2.5) () {$t=2$};

	\filldraw[draw=black, fill=blue] (4,0) rectangle node {} +(0.2,1);
	\filldraw[draw=black, fill=lightgray]  (4,1) rectangle node {} +(0.2,0.75);
	\node at (4, 2.5) () {$t=4$};

	\filldraw[draw=black, fill=blue] (6,0) rectangle node {} +(0.2,1);
	\filldraw[draw=black, fill=lightgray]  (6,1) rectangle node {} +(0.2,0.5);
	\node at (6, 2.5) () {$t=6$};

	\filldraw[draw=black, fill=blue] (8,0) rectangle node {} +(0.2,1);
	\filldraw[draw=black, fill=lightgray]  (8,1) rectangle node {} +(0.2,0.25);
	\node at (8, 2.5) () {$t=8$};

	\filldraw[draw=black, fill=blue] (10,0) rectangle node {} +(0.2,1);
	\filldraw[draw=black, fill=lightgray]  (10,1) rectangle node {} +(0.2,0.1);
	\node at (10, 2.5) () {$t=10$};

	% Legend
	\filldraw[draw=black, fill=blue] (7,-1) rectangle node {} +(0.2,0.2);
	\node at (8.3, -0.92) () {repayment};

	\filldraw[draw=black, fill=lightgray] (7,-1.5) rectangle node {} +(0.2,0.2);
	\node at (8, -1.44) () {coupon};

	\end{tikzpicture}
	\caption{Cashflow of a serial.}
	\label{fig:serialcf}
	\end{center}
	\end{figure}

\ctable[
	caption = Serial parameters,
	pos = ht,
	doinside=\small,
	center,
	]{llll}{
	}{
	\toprule
		Paramter & Type \\
	  \midrule
          Settlement & \texttt{Date} \\
          Maturity & \texttt{Date} \\
          Face & \texttt{Cash} \\
          Coupon rate & \texttt{Double} \\
          Settlements & \texttt{Settements} \\
          Day count convention & \texttt{(Daycount d) => d} \\
          Business day convention & \texttt{RollConvention} \\
	  \bottomrule
	}
	\begin{lstlisting}
	HQL> let settle = (read "2010-01-01")::Date
	HQL> let maturity1 = (read "2015-01-02")::Date
	HQL> let bullet  = Serial settle maturity (Cash 100 USD) 0.1 1 Following
	\end{lstlisting}

	\section{US Treasury Bonds}
	Treasury securities are the debt financing instruments of the United States federal government, and they are often referred to simply as Treasuries.
	\subsection{Treasury Bill - T-Bill}
	Short term debt instruments that mature in one year or less.
	\subsection{Treasury Note - T-Note}
	Medium term debt instruments that mature in two to ten years.
	\subsection{Treasury Bond - T-Bond}
	Long term debt instruments that mature in twenty to thirty years.

	\subsection{Example using HQL}
 The bullet bond is created through the interface with the constructor Bullet:

	\begin{lstlisting}
	HQL> let settle = (read "2010-01-01")::Date
	HQL> let maturity1 = (read "2015-01-02")::Date
	HQL> let bullet  = Bullet settle maturity (Cash 100 USD) 0.1 1 Following
	\end{lstlisting}
	
	Next, we'll calculate the present value of the previously defined bullet object. First we need a term structure. For more information about term structures in HQL see section xx below. The term structure below is a flat one, with a fixed rate of 5.0\%.
	
	\begin{lstlisting}
	HQL> let ts  = AnalyticalTermStructure (x = 5.0)	
	\end{lstlisting}

	\begin{lstlisting}	
	HQL> pv bullet ts
	\end{lstlisting}
	
	yields the following output:
	\vskip 0.5\FrameSep
	\begin{lstlisting}[style=Output]
	100.00
	\end{lstlisting}


	\framebox{TODO: Modify values for example}
%% ============================================================================================
%% Interest rate Swaps
%% ============================================================================================

	\subsection{Interest rate Swaps}
	Interest rate swaps are interest rate instruments what are priceable, and consist of two parts. Typically for a swap, the
	two parts, called legs, are the floating leg and the fixed leg. The floating leg is in fact, for a vanilla swap, the floating spot rate whereas the fixed leg is fixed interest rate specified in the contract. A swap pays an amount at predefined intervals, according to the fixed and floating leg. One party will receive the floating rate, and the other the fixed rate of the swap.

	\subsubsection{Theory and valuation}
	A swap is valuated as a portfolio of a zero coupon bond and a floating coupon bond. The amount (positive or negative) depends on the holds of the contract, hence the one valuating it.

	\subsubsection{Cash flows}
	\framebox{TODO: Illustrate cash flows of a swap using figure...}


%% ============================================================================================
%% Fitting the termstructure
%% ============================================================================================
\begin{comment}
	\section{Fitting the termstructure}
	Various methods of fitting a termstructure to market data (LIBOR, forex futures, bonds).

	\begin{lstlisting}
	HQL> let df*1000.0
	\end{lstlisting}
	yields the following output:
	\vskip 0.5\FrameSep
	\begin{lstlisting}[style=Output]
	775.6133702070988
	\end{lstlisting}
\end{comment}

\section{Options}
Options come in two variants, puts and calls. The call option gives the owner of the
option the right, but not the obligation, to buy the underlying asset at the strike price.
The put option gives the holder of the contract the right, but not the obligation, to sell
the underlying asset. The Black-Scholes formula describes the European option,
which can only be exercised on the maturity date, in contrast to, for example,
American options. The buyer of the option pays a premium for this in order to cover
the risk taken from the counterpart's side. Options have become very popular and they
are used in the major exchanges throughout the world, covering most asset classes \cite{astborg:fsharp}.
\\\\
The contract specifications for options will also depend on their type.
Generally, there are some properties that are more or less general to all of them.
The general specifications are as follows:

\begin{itemize}
\item \emph{Side}
\item \emph{Quantity}
\item \emph{Strike price}
\item \emph{Expiration date}
\item \emph{Settlement terms}
\end{itemize}

\subsection{European options}
European options are the basic form of options that other types of options extend.
American options and Exotic options are some examples. We'll stick to European
options in this section, which is the type HQL currently supports. HQL uses the closed
form Black-Scholes formula to price put and call options.
\\\\
The following are the assumptions made under the Black-Scholes formula \cite{astborg:fsharp}, derived in appendix X:
\begin{itemize}
\item No arbitrage
\item Possible to borrow money at a constant risk-free interest rate (throughout the
holding of the option)
\item Possible to buy, sell, and shortlist fractional amounts of underlying assets
\item No transaction costs
\item Price of the underlying asset follows a Brownian motion, constant drift,
and volatility
\item No dividends paid from underlying security
\end{itemize}

The famous Black-Scholes formula, with the parameters listed in table x, for the present value at time $t$.

\[ V(S,t) = \left\{
  \begin{array}{l l}
    C(S,t) = N(d_1)S-N(d_2)Ke^{-r(T-t)} & \quad \text{For call options} \\
    P(S,t) = N(-d_2)Ke^{-r(T-t)}-N(-d_1)S & \quad \text{Put option}
  \end{array} \right.\]

Where $d_1$ and $d_2$ are the two Black-Scholes parameters (input to the cumulative distrubution function $N$).

\begin{align} 
     d_1 &= \frac{1}{\sigma\sqrt{T - t}}\left[\ln\left(\frac{S}{K}\right) + \left(r + \frac{\sigma^2}{2}\right)(T - t)\right] \notag\\
     d_2 &= \frac{1}{\sigma\sqrt{T - t}}\left[\ln\left(\frac{S}{K}\right) + \left(r - \frac{\sigma^2}{2}\right)(T - t)\right] \notag\\
         &= d_1 - \sigma\sqrt{T - t} \notag
\end{align}

	\ctable[
	caption = Parameters used in Black-Scholes,
	pos = !hb,
	doinside=\small,
	center,
	]{llll}{
	}{
	\toprule
		Parameter & Description & Example\\
	  \midrule
              N & The cumulative distribution function & a \\
              T & Time to maturity, expressed in years & a\\
              S & The stock price or other underlying assets & a\\
              K & The strike price & a\\
              r & The risk-free annuallized interest rate & a\\

	  \bottomrule
	}
\vskip 1.5\FrameSep

\subsubsection{Pricing a call option}

        \begin{lstlisting}	
	HQL> let black_scholes Call 58.60 60.0 0.5 0.01 0.3
	\end{lstlisting}
yields the following output:
	\vskip 0.5\FrameSep
	\begin{lstlisting}[style=Output]
	4.465202269
	\end{lstlisting}


\subsubsection{Pricing a put option}
        \begin{lstlisting}	
	HQL> let black_scholes Put 58.60 60.0 0.5 0.01 0.3
	\end{lstlisting}
yields the following output:
	\vskip 0.5\FrameSep
	\begin{lstlisting}[style=Output]
	5.565951021
	\end{lstlisting}
%% ============================================================================================
%% The Portfolio
%% ============================================================================================

	\section{The Portfolio}
	A portfolio, is as the name suggests, a collection of position in financial instruments. Typically stocks, bonds, options, futures and so on. In this section we'll mainly discuss the portfolio supported in HQL. The portfolio is in itself priceable, which means one can get the present and future value of a portfolio, consisting of various instruments from the HQL library.

	The portfolio is also used as the building block for swaps. (to be continued...)


%% ============================================================================================
%% Day-count conventions
%% ============================================================================================
	\section{Day-count conventions}

	Day-count convention or year fraction, time between two dates t and T. Daycount conventions are defined in the module Utils.DayCount.

	\framebox{HQL example usage goes here}

	\ctable[
	caption = Input parameters to example,
	pos = ht,
	width = 80mm,
	center,
	doinside=\small
	]{lll}{
	}{
	  \toprule
		Convention & Time interval & $\delta$ \\
	  \midrule
		Actual/365 & Annually & 1.0 \\
		180/360 & Semi-annually & 0.5 \\
		90/360 & Quarterly & 0.25 \\
		30/360 & Monthly & 1/12 \\
	  \bottomrule
	}

	Time is measured in years. Day-count conventions describe the time measurement between two
	dates t and T. Dates are expressed by (day, month, year).

	\[ \tau = \boldsymbol{T} - \boldsymbol{t};  \left\{
	  \begin{array}{l l}
		  \boldsymbol{t} = (Y_1, M_1, D_1) \\
	\boldsymbol{T}=(Y_2,M_2,D_2)
	  \end{array} \right.\]


	\subsection{Actual/365}
	A year has 365 days, and the days are counted normally:
	\[
	\delta = \frac{\Delta \tau_D}{365} = \frac{D_2-D_1}{365} = \frac{\text{actual days between } \boldsymbol{t} \text{ and } \boldsymbol{T}}{365}
	\]

	\framebox{TODO: Add example}

	\subsection{Actual/360}
	Same as above except one year is said to be 360 days. Standard for US dollars
	\[
	\delta = \frac{\Delta \tau_D}{360} =\frac{D_2-D_1}{360} = \frac{\text{actual days between } \boldsymbol{t} \text{ and } \boldsymbol{T}}{360}
	\]

	\framebox{TODO: Add example}

	\subsection{30/360}
	Month are 30 days long, a year is 360 days long
	\[
	\delta = \frac{360 \cdot \Delta \tau_Y + 30 \cdot \Delta \tau_M + \Delta \tau_D}{360} =\frac{360(Y_2-Y_2)+30(M_2-M_1)+(D_2-D_1)}{360}
	\]

	\framebox{TODO: Add example}

	\subsection{Business day conventions}
	A business day convention is a convention for adjustment of dates when a specified date is not a good business day.
	\begin{description}
	  \item[Following] \hfill \\
	The adjusted date is the following good business day \ldots
	  \item[Preceding] \hfill \\
		The adjusted date is the preceding good business day \ldots
	  \item[Modified following] \hfill \\
	  The adjusted date is the following good business day unless the day is in the next calendar month \ldots
	  \item[End of month] \hfill \\
	Where the start date of a period is on the final business day of a particular calendar month \ldots
	\end{description}


        \section{Yahoo! Finance data}
        In this section we'll introduce the HQL functionality to download and parse data from Yahoo Finance\footnotemark[1].
        \fancyfootnotetext{1}{Yahoo! Finance - http://finance.yahoo.com}. The data at Yahoo Finance is available as CSV, comma separated values.
        
	\ctable[
	caption = Input parameters to web service,
	pos = !htp,
	width = 80mm,
	center,
	doinside=\small
	]{lll}{
	}{
	  \toprule
		Parameter & Description & Example \\
	  \midrule
                s & Instrument symbol & \string^{IRX} \\
                d & To month of year & 07 \\
                e & To day of month & 29 \\
                f & To year & 2014 \\
                a & From month of year & 00 \\
                b & From day of month & 1 \\
                c & From year & 2013 \\
	  \bottomrule
	}
        \noindent
        The data is mapped to a record when parsed, defined as follows:

%\begin{mdframed}
        \begin{minted}[tabsize=2,gobble=0,xleftmargin=25pt]{haskell}
data YahooEntry = YahooEntry
             { date :: LocalTime
             , open :: Double
             , high :: Double
             , low :: Double
             , close :: Double
             , volume :: Integer
             , adjclose :: Double
             } deriving (Eq, Show)
        \end{minted}
     %\end{mdframed}

        \subsection{Example - Downloading and parsing data}

        Let's look at an example where we use HQL to download and parse the daily prices for 13-week Treasury Bill (\string^{IRX}) from 2000-01-01 to 2014-01-01.
To do this, we use the command \texttt{wget} which is part of most Linux/Unix distributions:
        
        \begin{lstlisting}[breaklines=true]	
        ~/HQL$ wget http://ichart.finance.yahoo.com/table.csv?s= %5EIRX&a=00&b=1&c=2000&d=00&e=1&f=2014&g=d&ignore=.csv
	\end{lstlisting}

        This will download the data from Yahoo! Finance in the file \texttt{table.csv}. We'll now let HQL parse the file and plot the data using the
        visualization and plotting functionality, covered in more detail in the next section below.

        \framebox{TODO: HQL parsing and plot (code done)}

	\section{Visualization and plotting}
	As an extra feature, HQL supports visualization of cash flows in much the same fashion as the diagrams seen above for the various bond types.
	The visualization module is defined in \texttt{Utils.Graphics.Visualize module}.

	\begin{lstlisting}	
	HQL> let cash1 = Cash 500 USD
	HQL> let cash2 = Cash 1500 USD
	HQL> let sampleData = [("2014-06-01", cash1),("2015-01-01", cash1),("2015-06-01", cash1),("2016-01-01", cash1),("2016-06-01", cash2)]
	HQL> let GP.plotDefault $ cashflowDiagram testData	
	\end{lstlisting}

	\framebox{TODO: EPS plot output from HQL}


\newpage

%\input{Appendices/AppendixA}


\appendix

\section{Test cases} \label{app:testcases}

\subsection{Cross reference}
This section is a short cross reference between the test cases in Derivatives Expert (DE) and HQL. 

\ctable[
	caption = Cross reference for test cases and their respective outputs,
	pos = !htp,
	center,
	doinside=\small
	]{lllll}{
	}{
	  \toprule
		Test case DE & Test case HQL & Result DE & Result HQL & Explicit (exact) calculation\\
	  \midrule
                2 & - & 0.9203271932613013 & - & $(1+(5.6124/(100\cdot2)))^{-1.5\cdot2}$ \\
                9 & - & 0.7756133702070986 & - & $(1+8.5/(100\cdot12))^{-3\cdot12}$ \\
		11 & - & 0.7643885607510086 & - & $(1+(5 + (1/2)\cdot\sqrt{4.5})/(100\cdot2))^{-4.5\cdot2}$ \\
	        14 & - & 0.9592326139088729 & - & $(1+(8.5/(100\cdot2)))^{-0.5\cdot2}$ \\
	       	15 & - & 0.9593493353414723 & - & $e^{-8.3/100\cdot1/2}$ \\
	  \bottomrule
	}
\framebox{TODO: Map DE test cases to HQL}
\section{HQL formula sheet} \label{app:formulas}
In this section the formulas derived above and in this appendix, is summarized in a convinient table with descriptions.

\framebox{TODO: Table}

\section{Derivation of Black Scholes} \label{app:derivbs}
This section provides for a derivation of Black Scholes formula, following the method in the classical paper from 1973 \cite{blackscholes}.
\\\\
The following "ideal conditions" are assumed for the stock and the option \cite{blackscholes}\cite{astborg:fsharp}:
\begin{itemize}
\item No arbitrage
\item Possible to borrow money at a constant risk-free interest rate (throughout the
holding of the option)
\item Possible to buy, sell, and shortlist fractional amounts of underlying assets
\item No transaction costs
\item Price of the underlying asset follows a Brownian motion, constant drift,
and volatility
\item No dividends paid from underlying security
\end{itemize}

Under the assumptions above, the price of the stock depends only on the price of the underlying (for example stock), which follows
a geometric brownian motion. The geometric brownian motion is a modification to the original assumption made by Bachelier (Théorie de la speculation), where
the returns follow the brownian motions instead of the stock price itself \cite{bachelier}. The asset followes thus a geometrical brownian motion,

\[
\frac{dS}{S} = \mu dt + \sigma dW
\]

where $W$ is the Brownian motion, $\mu$ is the drift and $\sigma$ the annualized volatility, with normally distributed increments. Further, the payoff for the put and call option are know as,

\[ V(S,T) = \left\{
  \begin{array}{l l}
    max\left\{S_T-K,0 \right\} & \quad \text{For call options} \\
    max\left\{K-S_T,0 \right\} & \quad \text{Put option}
  \end{array} \right.\]

where $S_T$ is the price ot the asset at time $T$ and $K$ is the strike price of the option.

\[
dV = \left(\mu S \frac{\partial V}{\partial S} + \frac{\partial V}{\partial t} + \frac{1}{2}\sigma^2 S^2 \frac{\partial^2 V}{\partial S^2}\right)dt + \sigma S \frac{\partial V}{\partial S}\,dW
\]

\[ V(S,t) = \left\{
  \begin{array}{l l}
    C(S,t) = N(d_1)S-N(d_2)Ke^{-r(T-t)} & \quad \text{For call options} \\
    P(S,t) = N(-d_2)Ke^{-r(T-t)}-N(-d_1)S & \quad \text{Put option}
  \end{array} \right.\]

Where $d_1$ and $d_2$ are the two Black-Scholes parameters, as input to the cumulative distrubution function $N(p)$:

\begin{align} 
     d_1 &= \frac{1}{\sigma\sqrt{T - t}}\left[\ln\left(\frac{S}{K}\right) + \left(r + \frac{\sigma^2}{2}\right)(T - t)\right] \notag\\
     d_2 &= \frac{1}{\sigma\sqrt{T - t}}\left[\ln\left(\frac{S}{K}\right) + \left(r - \frac{\sigma^2}{2}\right)(T - t)\right] \notag\\
         &= d_1 - \sigma\sqrt{T - t} \notag
\end{align}

\gls{Settlement}
\gls{Maturity}
\gls{Period}
\gls{Basis}
\gls{EndMonthRule}
\gls{IssueDate}
\gls{Term}
\gls{Tenor}
\gls{Annuity}
\gls{Compounding}
\gls{Annualized_rate}
\gls{Amortized_bond}
\gls{Coupon_bond}
\gls{Zero_coupon_bond}
\gls{Serial}
\gls{Annuity}
\gls{Bullet}

The above result can be interpreted as follows. First, the stock price is
scaled using the cumulative distribution function with $d_1$ as a parameter.
Then, the stock price is reduced by the discounted strike price scaled by the cumulative
distribution function of $d_2$ . In other words, it's the difference between the stock price
and the strike price using probability scaling of each and discounting to present time.

\newpage
\printglossary[title=Terms and abbreviations,toctitle=Terms and abbreviations]

\newpage
%\addcontentsline{toc}{chapter}{\numberline{}Bibliography}
\bibliographystyle{acm}
\bibliography{hql}

\newpage
%\addcontentsline{toc}{chapter}{\numberline{}Index}
\printindex

\end{document}
